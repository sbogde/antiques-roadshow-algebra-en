% main.tex — English translation
\documentclass[12pt,a4paper]{book}

% Encoding and language
\usepackage[utf8]{inputenc}
\usepackage[T1]{fontenc}
\usepackage[english]{babel}

% Maths and layout
\usepackage{amsmath,amssymb,amsthm}
\DeclareMathOperator{\End}{End}
\DeclareMathOperator{\soc}{soc}
\DeclareMathOperator{\Rad}{Rad}

\usepackage{geometry}
\geometry{margin=3cm}
\usepackage{graphicx}
\usepackage{microtype}
\usepackage[hidelinks]{hyperref}

% ---------- Theorem environments (mirroring the RO version) ----------
\theoremstyle{plain}
% \newtheorem{theorem}{Theorem}[chapter]
\newtheorem{theorem}{Theorem}[section]   % [section] not [chapter]

\newtheorem{lemma}[theorem]{Lemma}
\newtheorem{proposition}[theorem]{Proposition}
\newtheorem{corollary}[theorem]{Corollary}

\theoremstyle{definition}
\newtheorem{definition}[theorem]{Definition}
\newtheorem{example}[theorem]{Example}

\theoremstyle{remark}
\newtheorem{remark}[theorem]{Remark}

% ---------- Title page ----------
\title{The Uniform (Co-Irreducible) Dimension of Rings and Modules}
\author{Supervisor: Prof. Dr Tiberiu Dumitrescu\\[0.5em]
Student: Sorin Bogde}
\date{Original thesis: University of Bucharest, Faculty of Mathematics, 1999\\
English translation: \today}

% \date{Original: 1999 \\ Translation: 2025}

\begin{document}

\frontmatter
\maketitle
\tableofcontents

% \chapter*{Preface}
% A short translator's preface can be added here later.

\mainmatter

\setcounter{chapter}{-1}
\chapter{Generalities}

\section{Semisimple modules}

\begin{definition}
A non-zero \(R\)-module \(S\) is called \emph{simple} if its only submodules are
\(0\) and \(S\).
\end{definition}

\begin{proposition}
Let \(S\) be an \(R\)-module. The following statements are equivalent:
\begin{enumerate}
  \item \(S\) is simple;
  \item for every non-zero element \(x \in S\) we have \(S = xR\);
  \item \(S \cong R/I\), where \(I\) is a maximal right ideal.
\end{enumerate}
\end{proposition}

\begin{lemma}[Schur]
Let \(S\) and \(S'\) be simple \(R\)-modules and let \(f\colon S \to S'\) be a homomorphism
of \(R\)-modules. Then \(f = 0\) or \(f\) is an isomorphism. In particular
\(\End_R(S)\) is a division ring.
\end{lemma}

\begin{definition}
Let \(M\) be an \(R\)-module and let \((S_i)_{i\in I}\) be the family of all simple
submodules of \(M\). If \(M = \sum_{i\in I} S_i\), then \(M\) is called \emph{semisimple}.
\end{definition}

\begin{proposition}
Let \(M\) be a semisimple \(R\)-module and \(N\) a submodule of \(M\).
Then there exists a subset \(J \subseteq I\) such that
\begin{enumerate}
  \item the family \((S_j)_{j\in J}\) is independent;
  \item \(M = N \oplus \bigl(\bigoplus_{j\in J} S_j\bigr)\).
\end{enumerate}
\end{proposition}

\begin{corollary}
With the above notation, for the semisimple module \(M\) there exists \(J \subseteq I\) such
that the family \((S_j)_{j\in J}\) is independent and
\[
  M = \bigoplus_{i\in I} M_i.
\]
\end{corollary}

\begin{corollary}
If \(M\) is a semisimple \(R\)-module and \(N\) a submodule of \(M\), then both \(N\) and
\(M/N\) are semisimple.
\end{corollary}

\begin{corollary}
A direct sum of semisimple modules is a semisimple module.
\end{corollary}

\begin{theorem}
Let \(M\) be an \(R\)-module. The following statements are equivalent:
\begin{enumerate}
  \item \(M\) is semisimple;
  \item \(M\) is isomorphic to a direct sum of simple modules;
  \item every submodule of \(M\) is a direct summand of \(M\);
  \item every short exact sequence
  \[
    0 \longrightarrow M' \xrightarrow{\,f\,} M \xrightarrow{\,g\,} M'' \longrightarrow 0
  \]
  splits.
\end{enumerate}
\end{theorem}

\begin{definition}
The sum of all simple submodules of \(M\) is called the \emph{socle} of \(M\) and is denoted
by \(\soc(M)\). If \(M\) has no simple submodule we put \(\soc(M) = 0\).
\end{definition}

\begin{proposition}
Let \(M\) and \(N\) be \(R\)-modules and \(f\colon M \to N\) a homomorphism.
Then \(f(\soc(M)) \subseteq \soc(N)\).
\end{proposition}

\begin{proposition}
Let \(M\) be an \(R\)-module and \(N\) a submodule of \(M\). Then
\[
  \soc(N) = \soc(M) \cap N.
\]
\end{proposition}

\begin{proposition}
If \(M = \bigoplus_{i\in I} M_i\), then
\[
  \soc(M) = \bigoplus_{i\in I} \soc(M_i).
\]
\end{proposition}

\begin{proposition}
Let \(R\) be a ring. Then the socle \(\soc(R_R)\) is a two-sided ideal of \(R\).
\end{proposition}



\section{Noetherian (Artinian) modules and Noetherian (Artinian) rings}

\begin{definition}
Let \(R\) be a ring and \(M\) a right \(R\)-module. We say that \(M\) satisfies
the \emph{maximal condition} (resp.\ the \emph{minimal condition}) if every
non-empty set of submodules of \(M\), ordered by inclusion, has a maximal
(resp.\ minimal) element.

We say that \(M\) satisfies the \emph{ascending} (resp.\ \emph{descending})
\emph{chain condition} if every ascending chain of submodules of \(M\)
\[
  M_1 \subseteq M_2 \subseteq \cdots \subseteq M_i \subseteq \cdots
\]
(resp.\ every descending chain
\[
  M_1 \supseteq M_2 \supseteq \cdots \supseteq M_i \supseteq \cdots
\])
is stationary, that is, there exists \(n \ge 1\) such that
\(M_n = M_{n+1} = \cdots\).
\end{definition}

\begin{proposition}
Let \(M\) be an \(R\)-module. The following statements are equivalent:
\begin{enumerate}
  \item \(M\) satisfies the maximal (minimal) condition;
  \item \(M\) satisfies the ascending (descending) chain condition.
\end{enumerate}
\end{proposition}

\begin{definition}
An \(R\)-module \(M\) is called \emph{noetherian} (resp.\ \emph{artinian}) if it
satisfies the maximal (resp.\ minimal) condition. The ring \(R\) is called
right noetherian (resp.\ right artinian) if the right module \(R_R\) is
noetherian (resp.\ artinian).
\end{definition}

\begin{example}
\leavevmode
\begin{enumerate}
  \item \(\mathbb{Z}\) is a noetherian ring but not artinian.
  \item Every finite group is a noetherian and artinian \(\mathbb{Z}\)-module.
  \item Every finite ring is noetherian and artinian.
  \item The ring \(\mathbb{Z}[X_1,X_2,\ldots,X_n,\ldots]\) is neither
        noetherian nor artinian:
        \[
          (X_1) \subsetneq (X_1,X_2) \subsetneq \cdots \subsetneq
          (X_1,\ldots,X_n) \subsetneq \cdots
        \]
        \[
          (X_1) \supsetneq (X_1^2) \supsetneq \cdots \supsetneq
          (X_1^k) \supsetneq \cdots
        \]
  \item The Prüfer \(p\)-group \(\mathbb{Z}_{p^\infty}\) is an artinian but not
        noetherian \(\mathbb{Z}\)-module.
\end{enumerate}
\end{example}

\begin{proposition}
Let \(N\) and \(P\) be submodules of \(M\) such that \(M = N + P\).
Then \(M\) is noetherian (artinian) if and only if \(N\) and \(P\) are
noetherian (artinian).
\end{proposition}

\begin{proposition}
For an \(R\)-module \(M\) the following statements are equivalent:
\begin{enumerate}
  \item \(M\) is noetherian;
  \item every submodule of \(M\) is finitely generated.
\end{enumerate}
\end{proposition}

\begin{proposition}
For an \(R\)-module \(M\) the following statements are equivalent:
\begin{enumerate}
  \item \(M\) is artinian;
  \item for every family \((X_i)_{i\in I}\) of submodules of \(M\) there exists
        a finite subset \(J \subseteq I\) such that
        \[
          \bigcap_{i\in I} X_i = \bigcap_{j\in J} X_j.
        \]
\end{enumerate}
\end{proposition}

\section{Finite length modules}

\begin{definition}
Let \(M\) be a non-zero right \(R\)-module. A \emph{composition series} or
\emph{Jordan--Hölder series} of \(M\) is a finite strictly ascending chain of
submodules
\[
  0 = X_0 \subset X_1 \subset \cdots \subset X_n = M
\]
such that \(X_{i+1}/X_i\) is a simple module for \(0 \le i \le n-1\).
The integer \(n\) is called the \emph{length} of the series, and the modules
\(X_{i+1}/X_i\) are called its \emph{factors}.
\end{definition}

\begin{proposition}
Let \(M\) be an \(R\)-module. The following statements are equivalent:
\begin{enumerate}
  \item \(M\) has a composition series;
  \item \(M\) is noetherian and artinian.
\end{enumerate}
\end{proposition}

\begin{proposition}
Let
\[
  0 \longrightarrow M' \longrightarrow M \longrightarrow M'' \longrightarrow 0
\]
be a short exact sequence of right \(R\)-modules. Then \(M\) has a composition
series if and only if both \(M'\) and \(M''\) have composition series.
\end{proposition}



If
\[
  0 = M_0 \subseteq M_1 \subseteq \cdots \subseteq M_n = M,\qquad
  0 = N_0 \subseteq N_1 \subseteq \cdots \subseteq N_p = M
\]
are two composition series of \(M\), we say that they are
\emph{equivalent} if \(n = p\) and there exists a bijection
\(\sigma : \{0,\ldots,n-1\} \to \{0,\ldots,n-1\}\) such that
\[
  M_{i+1}/M_i \cong M_{\sigma(i)+1}/M_{\sigma(i)}
  \quad (0 \le i \le n-1).
\]

\begin{theorem}[Jordan--Hölder]
If an \(R\)-module \(M\) has two composition series
\[
  0 = M_0 \subseteq M_1 \subseteq \cdots \subseteq M_n = M,\qquad
  0 = N_0 \subseteq N_1 \subseteq \cdots \subseteq N_p = M,
\]
then these two series are equivalent.
\end{theorem}

\begin{definition}
An \(R\)-module \(M\) which admits a composition series is called a
\emph{module of finite length}. The length of its composition series is
called the \emph{length} of \(M\) and is denoted by \(l(M)\). If \(M\)
admits no composition series, we say that \(M\) has \emph{infinite
length} and we write \(l(M)=\infty\).
\end{definition}

\begin{proposition}
Let
\[
  0 \longrightarrow M' \longrightarrow M \longrightarrow M'' \longrightarrow 0
\]
be a short exact sequence of \(R\)-modules of finite length. Then
\[
  l(M) = l(M') + l(M'').
\]
\end{proposition}

\begin{corollary}
Let \(M\) be an \(R\)-module of finite length and let \(N,L\) be
submodules of \(M\). Then:
\begin{enumerate}
  \item \(l(M) = l(N) + l(M/N)\);
  \item \(l(N+L) + l(N\cap L) = l(N) + l(L)\).
\end{enumerate}
\end{corollary}

\begin{corollary}
Let \(M\) be an \(R\)-module of finite length and let
\(M_1,M_2,\ldots,M_n\) be submodules of \(M\) such that
\[
  M = M_1 \oplus M_2 \oplus \cdots \oplus M_n.
\]
Then
\[
  l(M) = \sum_{i=1}^{n} l(M_i).
\]
\end{corollary}

\section{The Jacobson radical}

\medskip
\noindent\textbf{The Jacobson radical of a module.}

\begin{definition}
Let \(M\) be an \(R\)-module. The intersection of all maximal submodules
of \(M\) is called the \emph{Jacobson radical} of \(M\) and is denoted
by \(\Rad(M)\). If \(M\) has no maximal submodules, we adopt the
convention \(\Rad(M)=M\).
\end{definition}

\begin{remark}
If \(M\) is a finitely generated \(R\)-module, then \(\Rad(M)\ne M\).
\end{remark}

\begin{proposition}
Let \(M\) be an \(R\)-module. Then
\[
  \Rad(M)
  = \bigcap_{\substack{f : M \to S\\ S\ \text{simple}}} \ker(f)
  = \bigcap_{\substack{f : M \to X\\ X\ \text{semisimple}}} \ker(f).
\]
\end{proposition}

\begin{proposition}
Let \(f : M \to N\) be a homomorphism of \(R\)-modules. Then
\(f(\Rad(M)) \subseteq \Rad(N)\). If, in addition, \(f\) is an
epimorphism and \(\ker(f) \subseteq \Rad(M)\), then
\(f(\Rad(M)) = \Rad(N)\).
\end{proposition}

\begin{corollary}
For every \(R\)-module \(M\) we have \(\Rad(M/\Rad(M)) = 0\).
\end{corollary}

\begin{corollary}
If \(M\) is a semisimple \(R\)-module, then \(\Rad(M)=0\).
\end{corollary}

\begin{corollary}
If \(M = \bigoplus_{i\in I} M_i\), then
\[
  \Rad(M) = \bigoplus_{i\in I} \Rad(M_i).
\]
\end{corollary}

\begin{proposition}
Let \(M\) be an \(R\)-module with \(\Rad(M)\ne M\). Then
\[
  \Rad(M)
  = \bigcap \{\,L \le M \mid L \text{ is a superfluous submodule}\,\}.
\]
\end{proposition}

% Further chapters will be added as we progress.
% Example:
% \chapter{Title of Chapter 1}
% ...

\backmatter

% REFERENCES
% As with the RO version, we can either recreate the bibliography
% manually or from a .bib file.
% \bibliographystyle{plain}
% \bibliography{bibliography}

\end{document}
