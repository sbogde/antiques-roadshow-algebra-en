% main.tex — English translation
\documentclass[12pt,a4paper]{book}

% Encoding and language
\usepackage[utf8]{inputenc}
\usepackage[T1]{fontenc}
\usepackage[english]{babel}

% Maths and layout
\usepackage{amsmath,amssymb,amsthm}
\DeclareMathOperator{\End}{End}
\DeclareMathOperator{\soc}{soc}
\DeclareMathOperator{\Rad}{Rad}
\DeclareMathOperator{\Ker}{Ker}
\DeclareMathOperator{\Hom}{Hom}

% notations for essential submodule as per https://en.wikipedia.org/wiki/Essential_extension
\newcommand{\ess}{\trianglelefteq}  % Anderson–Fuller-style (default in text)
\newcommand{\esse}{\subseteq_{\!e}}      % Lam-style

\usepackage{geometry}
\geometry{margin=3cm}
\usepackage{graphicx}
\usepackage{microtype}
\usepackage[hidelinks]{hyperref}

% ---------- Theorem environments (mirroring the RO version) ----------
\theoremstyle{plain}
% \newtheorem{theorem}{Theorem}[section]   % [section] not [chapter]
\newtheorem{theorem}{Theorem}[chapter]
\newtheorem{lemma}[theorem]{Lemma}
\newtheorem{proposition}[theorem]{Proposition}
\newtheorem{corollary}[theorem]{Corollary}

\theoremstyle{definition}
\newtheorem{definition}[theorem]{Definition}
\newtheorem{example}[theorem]{Example}
\newtheorem*{example*}{Example}         % UNnumbered

\theoremstyle{remark}
\newtheorem{remark}[theorem]{Remark}

% ---------- Title page ----------
\title{The Uniform (Co-Irreducible) Dimension of Rings and Modules}
\author{Supervisor: Prof. Dr Tiberiu Dumitrescu\\[0.5em]
Student: Sorin Bogde}
\date{Original thesis: University of Bucharest, Faculty of Mathematics, 1999\\
English translation: \today}

% \date{Original: 1999 \\ Translation: 2025}

\begin{document}

\frontmatter
\maketitle
\tableofcontents

% \chapter*{Preface}
% A short translator's preface can be added here later.

\mainmatter

\setcounter{chapter}{-1}
\chapter{Generalities}

\section{Semisimple modules}

\begin{definition}
A non-zero \(R\)-module \(S\) is called \emph{simple} if its only submodules are
\(0\) and \(S\).
\end{definition}

\begin{proposition}
Let \(S\) be an \(R\)-module. The following statements are equivalent:
\begin{enumerate}
  \item \(S\) is simple;
  \item for every non-zero element \(x \in S\) we have \(S = xR\);
  \item \(S \cong R/I\), where \(I\) is a maximal right ideal.
\end{enumerate}
\end{proposition}

\begin{lemma}[Schur]
Let \(S\) and \(S'\) be simple \(R\)-modules and let \(f\colon S \to S'\) be a homomorphism
of \(R\)-modules. Then \(f = 0\) or \(f\) is an isomorphism. In particular
\(\End_R(S)\) is a division ring.
\end{lemma}

\begin{definition}
Let \(M\) be an \(R\)-module and let \((S_i)_{i\in I}\) be the family of all simple
submodules of \(M\). If \(M = \sum_{i\in I} S_i\), then \(M\) is called \emph{semisimple}.
\end{definition}

\begin{proposition}
Let \(M\) be a semisimple \(R\)-module and \(N\) a submodule of \(M\).
Then there exists a subset \(J \subseteq I\) such that
\begin{enumerate}
  \item the family \((S_j)_{j\in J}\) is independent;
  \item \(M = N \oplus \bigl(\bigoplus_{j\in J} S_j\bigr)\).
\end{enumerate}
\end{proposition}

\begin{corollary}
With the above notation, for the semisimple module \(M\) there exists \(J \subseteq I\) such
that the family \((S_j)_{j\in J}\) is independent and
\[
  M = \bigoplus_{i\in I} M_i.
\]
\end{corollary}

\begin{corollary}
If \(M\) is a semisimple \(R\)-module and \(N\) a submodule of \(M\), then both \(N\) and
\(M/N\) are semisimple.
\end{corollary}

\begin{corollary}
A direct sum of semisimple modules is a semisimple module.
\end{corollary}

\begin{theorem}
Let \(M\) be an \(R\)-module. The following statements are equivalent:
\begin{enumerate}
  \item \(M\) is semisimple;
  \item \(M\) is isomorphic to a direct sum of simple modules;
  \item every submodule of \(M\) is a direct summand of \(M\);
  \item every short exact sequence
  \[
    0 \longrightarrow M' \xrightarrow{\,f\,} M \xrightarrow{\,g\,} M'' \longrightarrow 0
  \]
  splits.
\end{enumerate}
\end{theorem}

\begin{definition}
The sum of all simple submodules of \(M\) is called the \emph{socle} of \(M\) and is denoted
by \(\soc(M)\). If \(M\) has no simple submodule we put \(\soc(M) = 0\).
\end{definition}

\begin{proposition}
Let \(M\) and \(N\) be \(R\)-modules and \(f\colon M \to N\) a homomorphism.
Then \(f(\soc(M)) \subseteq \soc(N)\).
\end{proposition}

\begin{proposition}
Let \(M\) be an \(R\)-module and \(N\) a submodule of \(M\). Then
\[
  \soc(N) = \soc(M) \cap N.
\]
\end{proposition}

\begin{proposition}
If \(M = \bigoplus_{i\in I} M_i\), then
\[
  \soc(M) = \bigoplus_{i\in I} \soc(M_i).
\]
\end{proposition}

\begin{proposition}
Let \(R\) be a ring. Then the socle \(\soc(R_R)\) is a two-sided ideal of \(R\).
\end{proposition}



\section{Noetherian (Artinian) modules and Noetherian (Artinian) rings}

\begin{definition}
Let \(R\) be a ring and \(M\) a right \(R\)-module. We say that \(M\) satisfies
the \emph{maximal condition} (resp.\ the \emph{minimal condition}) if every
non-empty set of submodules of \(M\), ordered by inclusion, has a maximal
(resp.\ minimal) element.

We say that \(M\) satisfies the \emph{ascending} (resp.\ \emph{descending})
\emph{chain condition} if every ascending chain of submodules of \(M\)
\[
  M_1 \subseteq M_2 \subseteq \cdots \subseteq M_i \subseteq \cdots
\]
(resp.\ every descending chain
\[
  M_1 \supseteq M_2 \supseteq \cdots \supseteq M_i \supseteq \cdots
\])
is stationary, that is, there exists \(n \ge 1\) such that
\(M_n = M_{n+1} = \cdots\).
\end{definition}

\begin{proposition}
Let \(M\) be an \(R\)-module. The following statements are equivalent:
\begin{enumerate}
  \item \(M\) satisfies the maximal (minimal) condition;
  \item \(M\) satisfies the ascending (descending) chain condition.
\end{enumerate}
\end{proposition}

\begin{definition}
An \(R\)-module \(M\) is called \emph{noetherian} (resp.\ \emph{artinian}) if it
satisfies the maximal (resp.\ minimal) condition. The ring \(R\) is called
right noetherian (resp.\ right artinian) if the right module \(R_R\) is
noetherian (resp.\ artinian).
\end{definition}

\begin{example}
\leavevmode
\begin{enumerate}
  \item \(\mathbb{Z}\) is a noetherian ring but not artinian.
  \item Every finite group is a noetherian and artinian \(\mathbb{Z}\)-module.
  \item Every finite ring is noetherian and artinian.
  \item The ring \(\mathbb{Z}[X_1,X_2,\ldots,X_n,\ldots]\) is neither
        noetherian nor artinian:
        \[
          (X_1) \subsetneq (X_1,X_2) \subsetneq \cdots \subsetneq
          (X_1,\ldots,X_n) \subsetneq \cdots
        \]
        \[
          (X_1) \supsetneq (X_1^2) \supsetneq \cdots \supsetneq
          (X_1^k) \supsetneq \cdots
        \]
  \item The Prüfer \(p\)-group \(\mathbb{Z}_{p^\infty}\) is an artinian but not
        noetherian \(\mathbb{Z}\)-module.
\end{enumerate}
\end{example}

\begin{proposition}
Let \(N\) and \(P\) be submodules of \(M\) such that \(M = N + P\).
Then \(M\) is noetherian (artinian) if and only if \(N\) and \(P\) are
noetherian (artinian).
\end{proposition}

\begin{proposition}
For an \(R\)-module \(M\) the following statements are equivalent:
\begin{enumerate}
  \item \(M\) is noetherian;
  \item every submodule of \(M\) is finitely generated.
\end{enumerate}
\end{proposition}

\begin{proposition}
For an \(R\)-module \(M\) the following statements are equivalent:
\begin{enumerate}
  \item \(M\) is artinian;
  \item for every family \((X_i)_{i\in I}\) of submodules of \(M\) there exists
        a finite subset \(J \subseteq I\) such that
        \[
          \bigcap_{i\in I} X_i = \bigcap_{j\in J} X_j.
        \]
\end{enumerate}
\end{proposition}

\section{Finite length modules}

\begin{definition}
Let \(M\) be a non-zero right \(R\)-module. A \emph{composition series} or
\emph{Jordan--Hölder series} of \(M\) is a finite strictly ascending chain of
submodules
\[
  0 = X_0 \subset X_1 \subset \cdots \subset X_n = M
\]
such that \(X_{i+1}/X_i\) is a simple module for \(0 \le i \le n-1\).
The integer \(n\) is called the \emph{length} of the series, and the modules
\(X_{i+1}/X_i\) are called its \emph{factors}.
\end{definition}

\begin{proposition}
Let \(M\) be an \(R\)-module. The following statements are equivalent:
\begin{enumerate}
  \item \(M\) has a composition series;
  \item \(M\) is noetherian and artinian.
\end{enumerate}
\end{proposition}

\begin{proposition}
Let
\[
  0 \longrightarrow M' \longrightarrow M \longrightarrow M'' \longrightarrow 0
\]
be a short exact sequence of right \(R\)-modules. Then \(M\) has a composition
series if and only if both \(M'\) and \(M''\) have composition series.
\end{proposition}



If
\[
  0 = M_0 \subseteq M_1 \subseteq \cdots \subseteq M_n = M,\qquad
  0 = N_0 \subseteq N_1 \subseteq \cdots \subseteq N_p = M
\]
are two composition series of \(M\), we say that they are
\emph{equivalent} if \(n = p\) and there exists a bijection
\(\sigma : \{0,\ldots,n-1\} \to \{0,\ldots,n-1\}\) such that
\[
  M_{i+1}/M_i \cong M_{\sigma(i)+1}/M_{\sigma(i)}
  \quad (0 \le i \le n-1).
\]

\begin{theorem}[Jordan--Hölder]
If an \(R\)-module \(M\) has two composition series
\[
  0 = M_0 \subseteq M_1 \subseteq \cdots \subseteq M_n = M,\qquad
  0 = N_0 \subseteq N_1 \subseteq \cdots \subseteq N_p = M,
\]
then these two series are equivalent.
\end{theorem}

\begin{definition}
An \(R\)-module \(M\) which admits a composition series is called a
\emph{module of finite length}. The length of its composition series is
called the \emph{length} of \(M\) and is denoted by \(l(M)\). If \(M\)
admits no composition series, we say that \(M\) has \emph{infinite
length} and we write \(l(M)=\infty\).
\end{definition}

\begin{proposition}
Let
\[
  0 \longrightarrow M' \longrightarrow M \longrightarrow M'' \longrightarrow 0
\]
be a short exact sequence of \(R\)-modules of finite length. Then
\[
  l(M) = l(M') + l(M'').
\]
\end{proposition}

\begin{corollary}
Let \(M\) be an \(R\)-module of finite length and let \(N,L\) be
submodules of \(M\). Then:
\begin{enumerate}
  \item \(l(M) = l(N) + l(M/N)\);
  \item \(l(N+L) + l(N\cap L) = l(N) + l(L)\).
\end{enumerate}
\end{corollary}

\begin{corollary}
Let \(M\) be an \(R\)-module of finite length and let
\(M_1,M_2,\ldots,M_n\) be submodules of \(M\) such that
\[
  M = M_1 \oplus M_2 \oplus \cdots \oplus M_n.
\]
Then
\[
  l(M) = \sum_{i=1}^{n} l(M_i).
\]
\end{corollary}

\section{The Jacobson radical}
\subsection*{The Jacobson radical of a module}


\begin{definition}
Let \(M\) be an \(R\)-module. The intersection of all maximal submodules
of \(M\) is called the \emph{Jacobson radical} of \(M\) and is denoted
by \(\Rad(M)\). If \(M\) has no maximal submodules, we adopt the
convention \(\Rad(M)=M\).
\end{definition}

\begin{remark}
If \(M\) is a finitely generated \(R\)-module, then \(\Rad(M)\ne M\).
\end{remark}

\begin{proposition}
Let \(M\) be an \(R\)-module. Then
\[
  \Rad(M)
  = \bigcap_{\substack{f : M \to S\\ S\ \text{simple}}} \ker(f)
  = \bigcap_{\substack{f : M \to X\\ X\ \text{semisimple}}} \ker(f).
\]
\end{proposition}

\begin{proposition}
Let \(f : M \to N\) be a homomorphism of \(R\)-modules. Then
\(f(\Rad(M)) \subseteq \Rad(N)\). If, in addition, \(f\) is an
epimorphism and \(\ker(f) \subseteq \Rad(M)\), then
\(f(\Rad(M)) = \Rad(N)\).
\end{proposition}

\begin{corollary}
For every \(R\)-module \(M\) we have \(\Rad(M/\Rad(M)) = 0\).
\end{corollary}

\begin{corollary}
If \(M\) is a semisimple \(R\)-module, then \(\Rad(M)=0\).
\end{corollary}

\begin{corollary}
If \(M = \bigoplus_{i\in I} M_i\), then
\[
  \Rad(M) = \bigoplus_{i\in I} \Rad(M_i).
\]
\end{corollary}

\begin{proposition}
Let \(M\) be an \(R\)-module with \(\Rad(M)\ne M\). Then
\[
  \Rad(M)
  = \bigcap \{\,L \le M \mid L \text{ is a superfluous submodule}\,\}.
\]
\end{proposition}


\begin{proposition}[Nakayama's Lemma]
Let \(M\) be a finitely generated \(R\)-module and \(N\) a submodule of \(M\).
If \(N + \Rad(M) = M\), then \(N = M\). (In other words, \(\Rad(M)\) is the
largest superfluous submodule of \(M\).)
\end{proposition}

\subsection*{The Jacobson radical of a ring}

Let \(R\) be a ring. We consider the left ideal \(\Rad({}_R R)\), the
intersection of all maximal left ideals of \(R\), and the right ideal
\(\Rad(R_R)\), the intersection of all maximal right ideals of \(R\).

\begin{proposition}
\leavevmode
\begin{enumerate}
  \item \(\Rad(R_R)\) is a two-sided ideal.
  \item \(\Rad(R_R)
          = \{\,r \in R \mid 1-ar \in U(R)\ \text{for all } a \in R\,\}\).
  \item \(\Rad(R_R) = \Rad({}_R R)\).
\end{enumerate}
\end{proposition}

\begin{definition}
The two-sided ideal \(\Rad(R_R) = \Rad({}_R R)\) is called the
\emph{Jacobson radical} of the ring \(R\) and is denoted by \(\Rad(R)\).
\end{definition}

\begin{proposition}
\leavevmode
\begin{enumerate}
  \item If \(J\) is a left (resp.\ right or two-sided) ideal such that
        \(1-x\) is invertible for every \(x \in J\), then \(J \subseteq \Rad(R)\).
  \item If \(J\) is a nil left (resp.\ right or two-sided) ideal, then
        \(J \subseteq \Rad(R)\).
\end{enumerate}
\end{proposition}

\begin{proposition}
Let \(\varphi : R \to S\) be a surjective ring homomorphism.
Then \(\varphi(\Rad(R)) \subseteq \Rad(S)\). If \(\ker(\varphi)
\subseteq \Rad(R)\), then \(\varphi(\Rad(R)) = \Rad(S)\).
\end{proposition}

\begin{proposition}
If \((R_i)_{i\in I}\) is a family of rings, then
\[
  \Rad\Bigl(\prod_{i\in I} R_i\Bigr) = \prod_{i\in I} \Rad(R_i).
\]
\end{proposition}

\begin{proposition}
Let \(M\) be an \(R\)-module. Then \(M\,\Rad(R) \subseteq \Rad(M)\).
\end{proposition}

\begin{theorem}
If \(R\) is an artinian ring, then \(\Rad(R)\) is nilpotent.
\end{theorem}

\section{Semisimple rings}

\begin{theorem}
For a ring \(R\) the following statements are equivalent:
\begin{enumerate}
  \item every non-zero right \(R\)-module is semisimple;
  \item \(R\) as a right \(R\)-module is semisimple;
  \item \(R\) is artinian and \(\Rad(R)=0\).
\end{enumerate}
\end{theorem}

\begin{definition}
A ring \(R\) satisfying any (hence all) of the above conditions is called
a \emph{semisimple ring}.
\end{definition}

\begin{proposition}
Let \(R\) be a semisimple ring and \(M\) a non-zero \(R\)-module.
The following statements are equivalent:
\begin{enumerate}
  \item \(M\) has finite length;
  \item \(M\) is noetherian;
  \item \(M\) is artinian.
\end{enumerate}
\end{proposition}

\begin{theorem}
Let \(R\) be a right artinian ring and \(M\) a non-zero right \(R\)-module.
The following statements are equivalent:
\begin{enumerate}
  \item \(M\) has finite length;
  \item \(M\) is noetherian;
  \item \(M\) is artinian.
\end{enumerate}
\end{theorem}

\begin{corollary}[Hopkins]
A right artinian ring (respectively a left artinian ring) is right
noetherian (respectively left noetherian).
\end{corollary}




\chapter{Essential submodules}

\begin{definition}
Let \(M\) be a right \(R\)-module. A submodule \(N\) of \(M\) is called
\emph{essential} (or we say that \(M\) is an essential extension of \(N\))
if \(N \cap N' \ne 0\) for every non-zero submodule \(N'\) of \(M\).
In this case we write \(N \ess M_R\).

A homomorphism of right \(R\)-modules \(f : M \to N\) is called
\emph{essential} if \(\operatorname{Im} f\) is an essential submodule of
\(N\) (i.e.\ \(\operatorname{Im} f \ess N_R\)).
\end{definition}

\begin{example*}
\leavevmode
\begin{enumerate}
  \item \(n\mathbb{Z} \ess \mathbb{Z}_\mathbb{Z}\) for every \(n \ge 1\).
  \item Every submodule of the Prüfer group \(\mathbb{Z}_{p^\infty}\) is
        essential.
\end{enumerate}
\end{example*}

\begin{remark}
Let \(M\) be a right \(R\)-module and \(N\) a submodule of \(M\).
Then \(N \ess M_R\) if and only if for every \(x \in M\), \(x \ne 0\),
there exists \(r \in R\) such that \(xr \in N \setminus \{0\}\).
\end{remark}

\begin{proof}
“\(\Rightarrow\)” Let \(x \in M \setminus \{0\}\). Since
\(0 \ne xR \le M_R\) and \(N \ess M_R\), we have
\(xR \cap N \ne 0\); hence there exists \(r \in R\) with
\(xr \in N \setminus \{0\}\).

“\(\Leftarrow\)” Let \(N' \le M_R\), \(N' \ne 0\). For every
\(x \in N' \setminus \{0\}\) there exists \(r \in R\) such that
\(xr \in N \setminus \{0\}\), so \(N \cap N' \ne 0\).
Thus \(N \ess M_R\).
\end{proof}

\begin{definition}
A monomorphism of right \(R\)-modules
\(f : N_R \to M_R\) is called \emph{essential} if
\(\operatorname{Im} f \ess M_R\). It is immediate that, if \(N\) is a
submodule of \(M\), then the canonical inclusion
\(i_N : N \to M\) is an essential monomorphism if and only if
\(N \ess M_R\).
\end{definition}


\begin{proposition}
A monomorphism \(f : N_R \to M_R\) is essential if and only if for every
right \(R\)-module \(M'\) and every \(g \in \Hom(M,M')\), the fact that
\(g \circ f\) is a monomorphism implies that \(g\) is a monomorphism.
\end{proposition}


\begin{proof}
“\(\Rightarrow\)” Let \(g\) be as in the statement and suppose that
\(g \circ f\) is a monomorphism. Assume \(\Ker g \ne 0\). Take
\(x \in \Ker g \cap \operatorname{Im} f \setminus \{0\}\). Then
\(x = f(x')\) for some \(x' \in N\) and \(g(x)=0\), so
\(g(f(x')) = 0\). Since \(g \circ f\) is a monomorphism, it follows that
\(x' = 0\) and hence \(x = 0\), a contradiction.

“\(\Leftarrow\)” If \(f\) is not an essential monomorphism, then there
exists \(N' \le M_R\), \(N' \ne 0\), such that
\(N' \cap \operatorname{Im} f = 0\). Consider the canonical projection
\(\pi_{N'} : M \to M/N'\). If \(x \in \Ker(\pi_{N'} \circ f)\), then
\(f(x) \in N'\), so \(f(x) = 0\), hence \(x = 0\). Thus
\(\pi_{N'} \circ f\) is injective. By the hypothesis in the statement,
this implies that \(\pi_{N'}\) is injective, so \(N' = 0\), a
contradiction.
\end{proof}


\begin{corollary}
Let \(M\) be a right \(R\)-module and \(N \le M_R\).
Then the following statements are equivalent:
\begin{enumerate}
  \item \(N \ess M_R\);
  \item the inclusion \(i_N : N \to M\) is an essential monomorphism;
  \item for every \(f \in \Hom(M,M')\), with \(M'\) an arbitrary
        \(R\)-module, the fact that \(f \circ i_N\) is a monomorphism
        implies that \(f\) is a monomorphism.
\end{enumerate}
\end{corollary}

\begin{proposition}
Let \(f : N_R \to M_R\) and \(g : M_R \to P_R\) be monomorphisms.
Then \(g \circ f\) is essential if and only if both \(g\) and \(f\) are
essential.
\end{proposition}

\begin{proof}

“\(\Leftarrow\)” Let \(z \in P \setminus \{0\}\). Since \(g\) is
essential, there exists \(r \in R\) such that
\(zr \in \operatorname{Im} g \setminus \{0\}\). Thus there exists
\(y \in M \setminus \{0\}\) with \(zr = g(y)\). As \(f\) is essential,
there exists \(r' \in R\) such that \(y r' \in \mathrm{Im}\, f \setminus \{0\}\).
Hence there exists \(x \in N \setminus \{0\}\) with \(y r' = f(x)\).
But \(z r' = g(y) r' = g(y r') = g(f(x))\).
If \(z r' = 0\), then \(g(f(x)) = 0\), hence \(x = 0\), a contradiction.
Therefore \(z r' \in \mathrm{Im}(g \circ f)\) and \(z r' \neq 0\), which shows that
\(g \circ f\) is essential.

“\(\Rightarrow\)” Let \(y \in M \setminus \{0\}\).
Since \(g\) is a monomorphism, \(g(y) \neq 0\).
Thus there exists \(r \in R\) such that
\(g(yr) \in \mathrm{Im}\, g \setminus \{0\}\) and \(g(yr) \neq 0\).
Hence there exists \(x \in N \setminus \{0\}\) such that
\(g(yr) = g(f(x))\), therefore \(yr = f(x) \in \mathrm{Im}\, f\),
which shows that \(f\) is an essential monomorphism.

If \(z \in P \setminus \{0\}\), there exists \(r \in R\) such that
\(z r \in \mathrm{Im}(g \circ f)\) and \(z r \neq 0\).
Since \(\mathrm{Im}(g \circ f) \subseteq \mathrm{Im}\, g\),
we obtain \(z r \in \mathrm{Im}\, g\), so \(g\) is essential.
\end{proof}


\begin{proposition}
Let \(M\) be a right \(R\)-module and
\(L_1, L_2, \ldots, L_n\) submodules of \(M\).
Then:

1) \(\displaystyle \bigcap_{i=1}^n L_i\) is essential in \(M\) iff each \(L_i\) is essential in \(M\).

2) If \(L_1 \subseteq L_2\) and \(L_1\) is essential in \(M\), then \(L_2\) is essential in \(M\).
\end{proposition}

The proof is obvious.

\begin{proposition}
Let \(K\) and \(L\) be submodules of \(M\).

1) If \(K \subseteq L \subseteq M\), then \(K \, \ess \, M\) iff  
\(K \, \ess \, L\) and \(L \, \ess \, M\).

2) If \(h : K_R \to M_R\) is a module morphism and \(L \, \ess \, M\),
then \(h^{-1}(L) \, \ess \, K\).

3) If \(L_1, L_2 \leq M_R\) with \(K_1 \, \ess \, L_1\) and \(K_2 \, \ess \, L_2\),
then \(K_1 \cap K_2 \, \ess \, L_1 \cap L_2\).
\end{proposition}

\textbf{Proof.}

1) Apply 1.5 and 1.6.

2) Let \(U\) be a non-zero submodule of \(K\).

(i) If \(h(U) = 0\), then \(U \subseteq \ker h \subseteq h^{-1}(L)\),
so \(U \cap h^{-1}(L) \neq 0\).

(ii) If \(h(U) \neq 0\), then \(h(U) \cap L \neq 0\).
Hence there exists \(u \in U\) with \(h(u) \in L\), \(h(u) \neq 0\),
so \(u \in U \cap h^{-1}(L)\) and \(u \neq 0\).

Thus \(h^{-1}(L) \, \ess \, K\).

3) Let \(0 \neq X \leq L_1 \cap L_2\).
Then \(X \subseteq L_1\), so \(0 \neq X \cap K_1 \leq L_1\).
But \(X \subseteq L_2\) implies  
\(0 \neq (X \cap K_1) \cap L_2 = X \cap (K_1 \cap K_2)\).
Hence \(K_1 \cap K_2 \, \ess \, L_1 \cap L_2\).

\begin{proposition}
Let \((K_\lambda)_{\lambda \in \Lambda}\),
\((L_\lambda)_{\lambda \in \Lambda}\) be families of submodules of \(M\).
If \((K_\lambda)\) is independent in \(M\) and  
\(K_\lambda \, \ess \, L_\lambda\) for all \(\lambda \in \Lambda\),
then \((L_\lambda)\) is independent in \(M\)
and
\[
\left(\bigoplus_{\lambda \in \Lambda} K_\lambda\right)
   \, \ess \,
\left(\bigoplus_{\lambda \in \Lambda} L_\lambda\right).
\]
\end{proposition}


\textbf{Proof.}

Let \(K_1 \, \ess \, L_1\), \(K_2 \, \ess \, L_2\) with \(K_1 \cap K_2 = 0\).
By 1.8(3), \(0 \, \ess \, L_1 \cap L_2\), hence \(L_1 \cap L_2 = 0\).

Let  
\(\pi_1 : L_1 \oplus L_2 \to L_1\),
\(\pi_2 : L_1 \oplus L_2 \to L_2\)
be the canonical projections.
Since \(K_1 \, \ess \, L_1\) and \(K_2 \, \ess \, L_2\),

\[
\pi_1^{-1}(K_1)
  = K_1 \oplus 0 \, \ess \, L_1 \oplus L_2,
\]
and
\[
\pi_2^{-1}(K_2)
  = 0 \oplus K_2 \, \ess \, L_1 \oplus L_2.
\]

Hence
\[
K_1 \oplus K_2
  = \pi_1^{-1}(K_1) \cap \pi_2^{-1}(K_2)
  \, \ess \, L_1 \oplus L_2.
\]

Induction gives the finite case.
For the general case, let \(0 \neq m \in \oplus_{\lambda \in \Lambda} L_\lambda\).
Then \(m\) lies in a finite direct sum
\(\oplus_{\lambda \in \Lambda_0} L_\lambda\)
for some finite \(\Lambda_0 \subseteq \Lambda\).
Since
\((\oplus_{\lambda \in \Lambda_0} K_\lambda)
   \, \ess \,
(\oplus_{\lambda \in \Lambda_0} L_\lambda)\),
there exists \(r \in R\) with
\(rm \in (\oplus_{\lambda \in \Lambda_0} K_\lambda) \setminus \{0\}
\subseteq (\oplus_{\lambda \in \Lambda} K_\lambda)\).
Thus
\[
\left(\bigoplus_{\lambda \in \Lambda} K_\lambda\right)
   \, \ess \,
\left(\bigoplus_{\lambda \in \Lambda} L_\lambda\right).
\]

\begin{proposition}
Let \(N\) be a submodule of \(M\).
Then there exists a submodule \(Q\) with
\(N \subseteq Q \subseteq M\)
such that \(Q\) is a maximal essential extension of \(N\) inside \(M\).
\end{proposition}

\textbf{Proof.}

Let
\[
\mathfrak{S}
= \{ L \leq M \; ;\; N \subseteq L \subseteq M,\; N \, \ess \, L \},
\]
ordered by inclusion.
Clearly \(\mathfrak{S} \neq \varnothing\), since \(N \in \mathfrak{S}\).

Let \((L_\lambda)_{\lambda \in \Lambda}\) be a totally ordered family of elements of
\(\mathfrak{S}\) and put
\[
L := \bigcup_{\lambda \in \Lambda} L_\lambda.
\]
Clearly \(L \leq M_R\).

Let \(x \in L \setminus \{0\}\).
Then there exists \(\lambda_0 \in \Lambda\) with \(x \in L_{\lambda_0}\).
Since \(N\) is essential in \(L_{\lambda_0}\), there exists \(r \in R\) such that
\(xr \in N\) and \(xr \neq 0\), hence \(L\) is an essential extension of \(N\).
Thus \(\mathfrak{S}\) is inductive and, by Zorn's lemma, \(\mathfrak{S}\) admits a
maximal element \(Q\) which satisfies the required conditions.


\begin{definition}
Let \(M\) be a right \(R\)-module and \(N \leq M_R\).
A submodule \(K \leq M_R\) is called a \emph{complement} of \(N\) in \(M\) if
\(K\) is a maximal submodule of \(M\) with the property that \(K \cap N = 0\).
A submodule \(K \leq M_R\) is called a \emph{complement submodule} of \(M\)
if there exists \(N \leq M_R\) such that \(K\) is a complement of \(N\) in \(M\).
\end{definition}

\begin{remark}
The set
\[
\widetilde{\mathfrak{S}}
  = \{\,L \leq M_R \mid N \cap L = 0\,\}
\]
is inductive and, by applying Zorn's lemma, it follows that there exists a
complement of \(N\) in \(M\).
In particular, \(0\) and \(M\) are complement submodules of \(M\).
\end{remark}


\begin{proposition}
Let \(M_R\), \(N \leq M_R\) and \(K \leq M_R\), with \(K\) a complement of \(N\) in \(M\).
There exists a complement \(Q\) of \(K\) in \(M\) such that \(N \subseteq Q\).
Moreover, \(Q\) is a maximal essential extension of \(N\) in \(M\).
\end{proposition}

\begin{proof}
It is easy to see that the set
\[
\widetilde{\mathfrak{S}}
  = \{\,L \leq M_R \mid K \cap L = 0,\; N \subseteq L\,\}
\]
is inductive, and Zorn's lemma guarantees the existence of \(Q\).

Let \(L\) be a non-zero submodule of \(Q\) such that \(L \cap N = 0\).
Put \(K_1 = L + K\).
Clearly \(K \subseteq K_1\).
If \(x \in N \cap (L + K)\), then \(x = y + z\) with \(y \in L\), \(z \in K\).
But \(z = x - y \in Q\).
Since \(Q \cap K = 0\), we obtain \(z = 0\) and hence \(x = y\).
From \(L \cap N = 0\) it follows that \(x = y = 0\), and therefore
\(N \cap (L + K) = 0\), which contradicts the fact that \(K\) is a complement of
\(N\) in \(M\).
Thus \(L \cap N \neq 0\) for every \(0 \neq L \leq Q\), so \(Q\) is an essential
extension of \(N\).

Suppose that there exists \(Q' \leq M_R\) with \(N \ess Q'\) and \(Q \subsetneq Q'\).
Since \(Q'\) is a complement of \(K\), we have \(Q' \cap K \neq 0\).
But \(N \cap (Q' \cap K) = 0\) and
\(0 \neq Q' \cap K \leq Q'\), contradicting \(N \ess Q'\).
Hence \(Q\) is a maximal essential extension of \(N\) in \(M\).
\end{proof}


\begin{definition}
A submodule \(N\) of \(M_R\) is called \emph{closed} if \(N\) has no proper
(meaning different from \(N\)) essential extension in \(M\).
\end{definition}


\begin{corollary}
Let \(M_R\) be a right \(R\)-module.
The complement submodules of \(M\) coincide with the closed submodules of \(M\).
\end{corollary}

\begin{proof}
From 1.13 it follows immediately that every closed submodule of \(M\) is a
complement submodule of \(M\).

Conversely, let \(K\) be a complement submodule of \(M_R\).
Then there exists \(N \leq M_R\) such that \(K\) is a complement of \(N\) in \(M\).
Assume that \(K\) has a proper essential extension in \(M\); that is, there exists
\(K' \leq M_R\) with \(K \ess K'\) and \(K \subsetneq K'\).
By the maximality of \(K\) we have \(K' \cap N \neq 0\), and since \(K \ess K'\),
it follows that
\[
K \cap K' \cap N \neq 0,
\]
contradiction.
\end{proof}

\begin{corollary}
Let \(N\) be a submodule of \(M_R\).
If \(K\) is a complement of \(N\) in \(M\), then:
\begin{enumerate}
  \item \((N + K) \ess M_R\).
  \item The canonical morphism \(\pi_K \circ i_N : N \to M/K\) is an essential monomorphism.
\end{enumerate}
\end{corollary}

\begin{proof}
(1) Let \(x \in M \setminus \{0\}\).
If \(x \notin K\), then \(K + Rx \neq K\) and hence
\(N \cap (K + Rx) \neq 0\).
Let \(y \in N \cap (K + Rx)\), \(y \neq 0\).
There exist \(z \in K\) and \(r \in R\) such that \(y = z + rx\).
If \(rx = 0\), then \(y = z\) and, since \(N \cap K = 0\), we obtain \(y = 0\),
a contradiction.
Thus \(rx \neq 0\) and, because \(rx = y - z\), we have \(rx \in N + K\),
which shows that \((N + K) \ess M_R\).

(2) We have \(\mathrm{Im}(\pi_K \circ i_N) = (N + K)/K\).
Let \(L/K\) be a non-zero submodule of \(M/K\).
Then
\[
\frac{N + K}{K} \cap \frac{L}{K}
   = \frac{(N + K) \cap L}{K}
   = \frac{N \cap L + K}{K}.
\]
Since \(K\) is a complement of \(N\), we have \(N \cap L \neq 0\), and hence
\[
\frac{N \cap L + K}{K} \neq 0,
\]
which shows that \(\pi_K \circ i_N\) is an essential monomorphism.
\end{proof}


% Further chapters will be added as we progress.
% Example:
% \chapter{Title of Chapter 1}
% ...

\backmatter

% REFERENCES
% As with the RO version, we can either recreate the bibliography
% manually or from a .bib file.
% \bibliographystyle{plain}
% \bibliography{bibliography}

\end{document}
