% main.tex — English translation
\documentclass[12pt,a4paper]{book}

% Encoding and language
\usepackage[utf8]{inputenc}
\usepackage[T1]{fontenc}
\usepackage[english]{babel}
\usepackage{tikz-cd}

% Maths and layout
\usepackage{amsmath,amssymb,amsthm}
\DeclareMathOperator{\End}{End}
\DeclareMathOperator{\soc}{soc}
\DeclareMathOperator{\Rad}{Rad}
\DeclareMathOperator{\Ker}{Ker}
\DeclareMathOperator{\Hom}{Hom}

% notations for essential submodule as per https://en.wikipedia.org/wiki/Essential_extension
\newcommand{\ess}{\trianglelefteq}  % Anderson–Fuller-style (default in text)
\newcommand{\esse}{\subseteq_{\!e}}      % Lam-style

\usepackage{geometry}
\geometry{margin=3cm}
\usepackage{graphicx}
\usepackage{microtype}
\usepackage[hidelinks]{hyperref}

% ---------- Theorem environments (mirroring the RO version) ----------
\theoremstyle{plain}
\newtheorem{theorem}{Theorem}[section]   % [section] not [chapter]
% \newtheorem{theorem}{Theorem}[chapter]
\newtheorem{lemma}[theorem]{Lemma}
\newtheorem{proposition}[theorem]{Proposition}
\newtheorem{corollary}[theorem]{Corollary}

\theoremstyle{definition}
\newtheorem{definition}[theorem]{Definition}
\newtheorem{example}[theorem]{Example}
\newtheorem*{example*}{Example}         % UNnumbered

\theoremstyle{remark}
\newtheorem{remark}[theorem]{Remark}

% ---------- Title page ----------
\title{The Uniform (Co-Irreducible) Dimension of Rings and Modules}
\author{Supervisor: Prof. Dr Tiberiu Dumitrescu\\[0.5em]
Student: Sorin Bogde}
\date{Original thesis: University of Bucharest, Faculty of Mathematics, 1999\\
English translation: \today}

% \date{Original: 1999 \\ Translation: 2025}

\begin{document}

\frontmatter
\maketitle
\tableofcontents

% \chapter*{Preface}
% A short translator's preface can be added here later.

\mainmatter

\setcounter{chapter}{-1}
\chapter{Generalities}

\section{Semisimple modules}

\begin{definition}
A non-zero \(R\)-module \(S\) is called \emph{simple} if its only submodules are
\(0\) and \(S\).
\end{definition}

\begin{proposition}
Let \(S\) be an \(R\)-module. The following statements are equivalent:
\begin{enumerate}
  \item \(S\) is simple;
  \item for every non-zero element \(x \in S\) we have \(S = xR\);
  \item \(S \cong R/I\), where \(I\) is a maximal right ideal.
\end{enumerate}
\end{proposition}

\begin{lemma}[Schur]
Let \(S\) and \(S'\) be simple \(R\)-modules and let \(f\colon S \to S'\) be a homomorphism
of \(R\)-modules. Then \(f = 0\) or \(f\) is an isomorphism. In particular
\(\End_R(S)\) is a division ring.
\end{lemma}

\begin{definition}
Let \(M\) be an \(R\)-module and let \((S_i)_{i\in I}\) be the family of all simple
submodules of \(M\). If \(M = \sum_{i\in I} S_i\), then \(M\) is called \emph{semisimple}.
\end{definition}

\begin{proposition}
Let \(M\) be a semisimple \(R\)-module and \(N\) a submodule of \(M\).
Then there exists a subset \(J \subseteq I\) such that
\begin{enumerate}
  \item the family \((S_j)_{j\in J}\) is independent;
  \item \(M = N \oplus \bigl(\bigoplus_{j\in J} S_j\bigr)\).
\end{enumerate}
\end{proposition}

\begin{corollary}
With the above notation, for the semisimple module \(M\) there exists \(J \subseteq I\) such
that the family \((S_j)_{j\in J}\) is independent and
\[
  M = \bigoplus_{i\in I} M_i.
\]
\end{corollary}

\begin{corollary}
If \(M\) is a semisimple \(R\)-module and \(N\) a submodule of \(M\), then both \(N\) and
\(M/N\) are semisimple.
\end{corollary}

\begin{corollary}
A direct sum of semisimple modules is a semisimple module.
\end{corollary}

\begin{theorem}
Let \(M\) be an \(R\)-module. The following statements are equivalent:
\begin{enumerate}
  \item \(M\) is semisimple;
  \item \(M\) is isomorphic to a direct sum of simple modules;
  \item every submodule of \(M\) is a direct summand of \(M\);
  \item every short exact sequence
  \[
    0 \longrightarrow M' \xrightarrow{\,f\,} M \xrightarrow{\,g\,} M'' \longrightarrow 0
  \]
  splits.
\end{enumerate}
\end{theorem}

\begin{definition}
The sum of all simple submodules of \(M\) is called the \emph{socle} of \(M\) and is denoted
by \(\soc(M)\). If \(M\) has no simple submodule we put \(\soc(M) = 0\).
\end{definition}

\begin{proposition}
Let \(M\) and \(N\) be \(R\)-modules and \(f\colon M \to N\) a homomorphism.
Then \(f(\soc(M)) \subseteq \soc(N)\).
\end{proposition}

\begin{proposition}
Let \(M\) be an \(R\)-module and \(N\) a submodule of \(M\). Then
\[
  \soc(N) = \soc(M) \cap N.
\]
\end{proposition}

\begin{proposition}
If \(M = \bigoplus_{i\in I} M_i\), then
\[
  \soc(M) = \bigoplus_{i\in I} \soc(M_i).
\]
\end{proposition}

\begin{proposition}
Let \(R\) be a ring. Then the socle \(\soc(R_R)\) is a two-sided ideal of \(R\).
\end{proposition}



\section{Noetherian (Artinian) modules and Noetherian (Artinian) rings}

\begin{definition}
Let \(R\) be a ring and \(M\) a right \(R\)-module. We say that \(M\) satisfies
the \emph{maximal condition} (resp.\ the \emph{minimal condition}) if every
non-empty set of submodules of \(M\), ordered by inclusion, has a maximal
(resp.\ minimal) element.

We say that \(M\) satisfies the \emph{ascending} (resp.\ \emph{descending})
\emph{chain condition} if every ascending chain of submodules of \(M\)
\[
  M_1 \subseteq M_2 \subseteq \cdots \subseteq M_i \subseteq \cdots
\]
(resp.\ every descending chain
\[
  M_1 \supseteq M_2 \supseteq \cdots \supseteq M_i \supseteq \cdots
\])
is stationary, that is, there exists \(n \ge 1\) such that
\(M_n = M_{n+1} = \cdots\).
\end{definition}

\begin{proposition}
Let \(M\) be an \(R\)-module. The following statements are equivalent:
\begin{enumerate}
  \item \(M\) satisfies the maximal (minimal) condition;
  \item \(M\) satisfies the ascending (descending) chain condition.
\end{enumerate}
\end{proposition}

\begin{definition}
An \(R\)-module \(M\) is called \emph{noetherian} (resp.\ \emph{artinian}) if it
satisfies the maximal (resp.\ minimal) condition. The ring \(R\) is called
right noetherian (resp.\ right artinian) if the right module \(R_R\) is
noetherian (resp.\ artinian).
\end{definition}

\begin{example}
\leavevmode
\begin{enumerate}
  \item \(\mathbb{Z}\) is a noetherian ring but not artinian.
  \item Every finite group is a noetherian and artinian \(\mathbb{Z}\)-module.
  \item Every finite ring is noetherian and artinian.
  \item The ring \(\mathbb{Z}[X_1,X_2,\ldots,X_n,\ldots]\) is neither
        noetherian nor artinian:
        \[
          (X_1) \subsetneq (X_1,X_2) \subsetneq \cdots \subsetneq
          (X_1,\ldots,X_n) \subsetneq \cdots
        \]
        \[
          (X_1) \supsetneq (X_1^2) \supsetneq \cdots \supsetneq
          (X_1^k) \supsetneq \cdots
        \]
  \item The Prüfer \(p\)-group \(\mathbb{Z}_{p^\infty}\) is an artinian but not
        noetherian \(\mathbb{Z}\)-module.
\end{enumerate}
\end{example}

\begin{proposition}
Let \(N\) and \(P\) be submodules of \(M\) such that \(M = N + P\).
Then \(M\) is noetherian (artinian) if and only if \(N\) and \(P\) are
noetherian (artinian).
\end{proposition}

\begin{proposition}
For an \(R\)-module \(M\) the following statements are equivalent:
\begin{enumerate}
  \item \(M\) is noetherian;
  \item every submodule of \(M\) is finitely generated.
\end{enumerate}
\end{proposition}

\begin{proposition}
For an \(R\)-module \(M\) the following statements are equivalent:
\begin{enumerate}
  \item \(M\) is artinian;
  \item for every family \((X_i)_{i\in I}\) of submodules of \(M\) there exists
        a finite subset \(J \subseteq I\) such that
        \[
          \bigcap_{i\in I} X_i = \bigcap_{j\in J} X_j.
        \]
\end{enumerate}
\end{proposition}

\section{Finite length modules}

\begin{definition}
Let \(M\) be a non-zero right \(R\)-module. A \emph{composition series} or
\emph{Jordan--Hölder series} of \(M\) is a finite strictly ascending chain of
submodules
\[
  0 = X_0 \subset X_1 \subset \cdots \subset X_n = M
\]
such that \(X_{i+1}/X_i\) is a simple module for \(0 \le i \le n-1\).
The integer \(n\) is called the \emph{length} of the series, and the modules
\(X_{i+1}/X_i\) are called its \emph{factors}.
\end{definition}

\begin{proposition}
Let \(M\) be an \(R\)-module. The following statements are equivalent:
\begin{enumerate}
  \item \(M\) has a composition series;
  \item \(M\) is noetherian and artinian.
\end{enumerate}
\end{proposition}

\begin{proposition}
Let
\[
  0 \longrightarrow M' \longrightarrow M \longrightarrow M'' \longrightarrow 0
\]
be a short exact sequence of right \(R\)-modules. Then \(M\) has a composition
series if and only if both \(M'\) and \(M''\) have composition series.
\end{proposition}



If
\[
  0 = M_0 \subseteq M_1 \subseteq \cdots \subseteq M_n = M,\qquad
  0 = N_0 \subseteq N_1 \subseteq \cdots \subseteq N_p = M
\]
are two composition series of \(M\), we say that they are
\emph{equivalent} if \(n = p\) and there exists a bijection
\(\sigma : \{0,\ldots,n-1\} \to \{0,\ldots,n-1\}\) such that
\[
  M_{i+1}/M_i \cong M_{\sigma(i)+1}/M_{\sigma(i)}
  \quad (0 \le i \le n-1).
\]

\begin{theorem}[Jordan--Hölder]
If an \(R\)-module \(M\) has two composition series
\[
  0 = M_0 \subseteq M_1 \subseteq \cdots \subseteq M_n = M,\qquad
  0 = N_0 \subseteq N_1 \subseteq \cdots \subseteq N_p = M,
\]
then these two series are equivalent.
\end{theorem}

\begin{definition}
An \(R\)-module \(M\) which admits a composition series is called a
\emph{module of finite length}. The length of its composition series is
called the \emph{length} of \(M\) and is denoted by \(l(M)\). If \(M\)
admits no composition series, we say that \(M\) has \emph{infinite
length} and we write \(l(M)=\infty\).
\end{definition}

\begin{proposition}
Let
\[
  0 \longrightarrow M' \longrightarrow M \longrightarrow M'' \longrightarrow 0
\]
be a short exact sequence of \(R\)-modules of finite length. Then
\[
  l(M) = l(M') + l(M'').
\]
\end{proposition}

\begin{corollary}
Let \(M\) be an \(R\)-module of finite length and let \(N,L\) be
submodules of \(M\). Then:
\begin{enumerate}
  \item \(l(M) = l(N) + l(M/N)\);
  \item \(l(N+L) + l(N\cap L) = l(N) + l(L)\).
\end{enumerate}
\end{corollary}

\begin{corollary}
Let \(M\) be an \(R\)-module of finite length and let
\(M_1,M_2,\ldots,M_n\) be submodules of \(M\) such that
\[
  M = M_1 \oplus M_2 \oplus \cdots \oplus M_n.
\]
Then
\[
  l(M) = \sum_{i=1}^{n} l(M_i).
\]
\end{corollary}

\section{The Jacobson radical}
\subsection*{The Jacobson radical of a module}


\begin{definition}
Let \(M\) be an \(R\)-module. The intersection of all maximal submodules
of \(M\) is called the \emph{Jacobson radical} of \(M\) and is denoted
by \(\Rad(M)\). If \(M\) has no maximal submodules, we adopt the
convention \(\Rad(M)=M\).
\end{definition}

\begin{remark}
If \(M\) is a finitely generated \(R\)-module, then \(\Rad(M)\ne M\).
\end{remark}

\begin{proposition}
Let \(M\) be an \(R\)-module. Then
\[
  \Rad(M)
  = \bigcap_{\substack{f : M \to S\\ S\ \text{simple}}} \ker(f)
  = \bigcap_{\substack{f : M \to X\\ X\ \text{semisimple}}} \ker(f).
\]
\end{proposition}

\begin{proposition}
Let \(f : M \to N\) be a homomorphism of \(R\)-modules. Then
\(f(\Rad(M)) \subseteq \Rad(N)\). If, in addition, \(f\) is an
epimorphism and \(\ker(f) \subseteq \Rad(M)\), then
\(f(\Rad(M)) = \Rad(N)\).
\end{proposition}

\begin{corollary}
For every \(R\)-module \(M\) we have \(\Rad(M/\Rad(M)) = 0\).
\end{corollary}

\begin{corollary}
If \(M\) is a semisimple \(R\)-module, then \(\Rad(M)=0\).
\end{corollary}

\begin{corollary}
If \(M = \bigoplus_{i\in I} M_i\), then
\[
  \Rad(M) = \bigoplus_{i\in I} \Rad(M_i).
\]
\end{corollary}

\begin{proposition}
Let \(M\) be an \(R\)-module with \(\Rad(M)\ne M\). Then
\[
  \Rad(M)
  = \bigcap \{\,L \le M \mid L \text{ is a superfluous submodule}\,\}.
\]
\end{proposition}


\begin{proposition}[Nakayama's Lemma]
Let \(M\) be a finitely generated \(R\)-module and \(N\) a submodule of \(M\).
If \(N + \Rad(M) = M\), then \(N = M\). (In other words, \(\Rad(M)\) is the
largest superfluous submodule of \(M\).)
\end{proposition}

\subsection*{The Jacobson radical of a ring}

Let \(R\) be a ring. We consider the left ideal \(\Rad({}_R R)\), the
intersection of all maximal left ideals of \(R\), and the right ideal
\(\Rad(R_R)\), the intersection of all maximal right ideals of \(R\).

\begin{proposition}
\leavevmode
\begin{enumerate}
  \item \(\Rad(R_R)\) is a two-sided ideal.
  \item \(\Rad(R_R)
          = \{\,r \in R \mid 1-ar \in U(R)\ \text{for all } a \in R\,\}\).
  \item \(\Rad(R_R) = \Rad({}_R R)\).
\end{enumerate}
\end{proposition}

\begin{definition}
The two-sided ideal \(\Rad(R_R) = \Rad({}_R R)\) is called the
\emph{Jacobson radical} of the ring \(R\) and is denoted by \(\Rad(R)\).
\end{definition}

\begin{proposition}
\leavevmode
\begin{enumerate}
  \item If \(J\) is a left (resp.\ right or two-sided) ideal such that
        \(1-x\) is invertible for every \(x \in J\), then \(J \subseteq \Rad(R)\).
  \item If \(J\) is a nil left (resp.\ right or two-sided) ideal, then
        \(J \subseteq \Rad(R)\).
\end{enumerate}
\end{proposition}

\begin{proposition}
Let \(\varphi : R \to S\) be a surjective ring homomorphism.
Then \(\varphi(\Rad(R)) \subseteq \Rad(S)\). If \(\ker(\varphi)
\subseteq \Rad(R)\), then \(\varphi(\Rad(R)) = \Rad(S)\).
\end{proposition}

\begin{proposition}
If \((R_i)_{i\in I}\) is a family of rings, then
\[
  \Rad\Bigl(\prod_{i\in I} R_i\Bigr) = \prod_{i\in I} \Rad(R_i).
\]
\end{proposition}

\begin{proposition}
Let \(M\) be an \(R\)-module. Then \(M\,\Rad(R) \subseteq \Rad(M)\).
\end{proposition}

\begin{theorem}
If \(R\) is an artinian ring, then \(\Rad(R)\) is nilpotent.
\end{theorem}

\section{Semisimple rings}

\begin{theorem}
For a ring \(R\) the following statements are equivalent:
\begin{enumerate}
  \item every non-zero right \(R\)-module is semisimple;
  \item \(R\) as a right \(R\)-module is semisimple;
  \item \(R\) is artinian and \(\Rad(R)=0\).
\end{enumerate}
\end{theorem}

\begin{definition}
A ring \(R\) satisfying any (hence all) of the above conditions is called
a \emph{semisimple ring}.
\end{definition}

\begin{proposition}
Let \(R\) be a semisimple ring and \(M\) a non-zero \(R\)-module.
The following statements are equivalent:
\begin{enumerate}
  \item \(M\) has finite length;
  \item \(M\) is noetherian;
  \item \(M\) is artinian.
\end{enumerate}
\end{proposition}

\begin{theorem}
Let \(R\) be a right artinian ring and \(M\) a non-zero right \(R\)-module.
The following statements are equivalent:
\begin{enumerate}
  \item \(M\) has finite length;
  \item \(M\) is noetherian;
  \item \(M\) is artinian.
\end{enumerate}
\end{theorem}

\begin{corollary}[Hopkins]
A right artinian ring (respectively a left artinian ring) is right
noetherian (respectively left noetherian).
\end{corollary}




\chapter{Essential submodules}

\begin{definition}
Let \(M\) be a right \(R\)-module. A submodule \(N\) of \(M\) is called
\emph{essential} (or we say that \(M\) is an essential extension of \(N\))
if \(N \cap N' \ne 0\) for every non-zero submodule \(N'\) of \(M\).
In this case we write \(N \ess M_R\).

A homomorphism of right \(R\)-modules \(f : M \to N\) is called
\emph{essential} if \(\operatorname{Im} f\) is an essential submodule of
\(N\) (i.e.\ \(\operatorname{Im} f \ess N_R\)).
\end{definition}

\begin{example*}
\leavevmode
\begin{enumerate}
  \item \(n\mathbb{Z} \ess \mathbb{Z}_\mathbb{Z}\) for every \(n \ge 1\).
  \item Every submodule of the Prüfer group \(\mathbb{Z}_{p^\infty}\) is
        essential.
\end{enumerate}
\end{example*}

\begin{remark}
Let \(M\) be a right \(R\)-module and \(N\) a submodule of \(M\).
Then \(N \ess M_R\) if and only if for every \(x \in M\), \(x \ne 0\),
there exists \(r \in R\) such that \(xr \in N \setminus \{0\}\).
\end{remark}

\begin{proof}
“\(\Rightarrow\)” Let \(x \in M \setminus \{0\}\). Since
\(0 \ne xR \le M_R\) and \(N \ess M_R\), we have
\(xR \cap N \ne 0\); hence there exists \(r \in R\) with
\(xr \in N \setminus \{0\}\).

“\(\Leftarrow\)” Let \(N' \le M_R\), \(N' \ne 0\). For every
\(x \in N' \setminus \{0\}\) there exists \(r \in R\) such that
\(xr \in N \setminus \{0\}\), so \(N \cap N' \ne 0\).
Thus \(N \ess M_R\).
\end{proof}

\begin{definition}
A monomorphism of right \(R\)-modules
\(f : N_R \to M_R\) is called \emph{essential} if
\(\operatorname{Im} f \ess M_R\). It is immediate that, if \(N\) is a
submodule of \(M\), then the canonical inclusion
\(i_N : N \to M\) is an essential monomorphism if and only if
\(N \ess M_R\).
\end{definition}


\begin{proposition}
A monomorphism \(f : N_R \to M_R\) is essential if and only if for every
right \(R\)-module \(M'\) and every \(g \in \Hom(M,M')\), the fact that
\(g \circ f\) is a monomorphism implies that \(g\) is a monomorphism.
\end{proposition}


\begin{proof}
“\(\Rightarrow\)” Let \(g\) be as in the statement and suppose that
\(g \circ f\) is a monomorphism. Assume \(\Ker g \ne 0\). Take
\(x \in \Ker g \cap \operatorname{Im} f \setminus \{0\}\). Then
\(x = f(x')\) for some \(x' \in N\) and \(g(x)=0\), so
\(g(f(x')) = 0\). Since \(g \circ f\) is a monomorphism, it follows that
\(x' = 0\) and hence \(x = 0\), a contradiction.

“\(\Leftarrow\)” If \(f\) is not an essential monomorphism, then there
exists \(N' \le M_R\), \(N' \ne 0\), such that
\(N' \cap \operatorname{Im} f = 0\). Consider the canonical projection
\(\pi_{N'} : M \to M/N'\). If \(x \in \Ker(\pi_{N'} \circ f)\), then
\(f(x) \in N'\), so \(f(x) = 0\), hence \(x = 0\). Thus
\(\pi_{N'} \circ f\) is injective. By the hypothesis in the statement,
this implies that \(\pi_{N'}\) is injective, so \(N' = 0\), a
contradiction.
\end{proof}


\begin{corollary}
Let \(M\) be a right \(R\)-module and \(N \le M_R\).
Then the following statements are equivalent:
\begin{enumerate}
  \item \(N \ess M_R\);
  \item the inclusion \(i_N : N \to M\) is an essential monomorphism;
  \item for every \(f \in \Hom(M,M')\), with \(M'\) an arbitrary
        \(R\)-module, the fact that \(f \circ i_N\) is a monomorphism
        implies that \(f\) is a monomorphism.
\end{enumerate}
\end{corollary}

\begin{proposition}
Let \(f : N_R \to M_R\) and \(g : M_R \to P_R\) be monomorphisms.
Then \(g \circ f\) is essential if and only if both \(g\) and \(f\) are
essential.
\end{proposition}

\begin{proof}

“\(\Leftarrow\)” Let \(z \in P \setminus \{0\}\). Since \(g\) is
essential, there exists \(r \in R\) such that
\(zr \in \operatorname{Im} g \setminus \{0\}\). Thus there exists
\(y \in M \setminus \{0\}\) with \(zr = g(y)\). As \(f\) is essential,
there exists \(r' \in R\) such that \(y r' \in \mathrm{Im}\, f \setminus \{0\}\).
Hence there exists \(x \in N \setminus \{0\}\) with \(y r' = f(x)\).
But \(z r' = g(y) r' = g(y r') = g(f(x))\).
If \(z r' = 0\), then \(g(f(x)) = 0\), hence \(x = 0\), a contradiction.
Therefore \(z r' \in \mathrm{Im}(g \circ f)\) and \(z r' \neq 0\), which shows that
\(g \circ f\) is essential.

“\(\Rightarrow\)” Let \(y \in M \setminus \{0\}\).
Since \(g\) is a monomorphism, \(g(y) \neq 0\).
Thus there exists \(r \in R\) such that
\(g(yr) \in \mathrm{Im}\, g \setminus \{0\}\) and \(g(yr) \neq 0\).
Hence there exists \(x \in N \setminus \{0\}\) such that
\(g(yr) = g(f(x))\), therefore \(yr = f(x) \in \mathrm{Im}\, f\),
which shows that \(f\) is an essential monomorphism.

If \(z \in P \setminus \{0\}\), there exists \(r \in R\) such that
\(z r \in \mathrm{Im}(g \circ f)\) and \(z r \neq 0\).
Since \(\mathrm{Im}(g \circ f) \subseteq \mathrm{Im}\, g\),
we obtain \(z r \in \mathrm{Im}\, g\), so \(g\) is essential.
\end{proof}


\begin{proposition}
Let \(M\) be a right \(R\)-module and
\(L_1, L_2, \ldots, L_n\) submodules of \(M\).
Then:

1) \(\displaystyle \bigcap_{i=1}^n L_i\) is essential in \(M\) iff each \(L_i\) is essential in \(M\).

2) If \(L_1 \subseteq L_2\) and \(L_1\) is essential in \(M\), then \(L_2\) is essential in \(M\).
\end{proposition}

The proof is obvious.

\begin{proposition}
Let \(K\) and \(L\) be submodules of \(M\).

1) If \(K \subseteq L \subseteq M\), then \(K \, \ess \, M\) iff  
\(K \, \ess \, L\) and \(L \, \ess \, M\).

2) If \(h : K_R \to M_R\) is a module morphism and \(L \, \ess \, M\),
then \(h^{-1}(L) \, \ess \, K\).

3) If \(L_1, L_2 \leq M_R\) with \(K_1 \, \ess \, L_1\) and \(K_2 \, \ess \, L_2\),
then \(K_1 \cap K_2 \, \ess \, L_1 \cap L_2\).
\end{proposition}

\textbf{Proof.}

1) Apply 1.5 and 1.6.

2) Let \(U\) be a non-zero submodule of \(K\).

(i) If \(h(U) = 0\), then \(U \subseteq \ker h \subseteq h^{-1}(L)\),
so \(U \cap h^{-1}(L) \neq 0\).

(ii) If \(h(U) \neq 0\), then \(h(U) \cap L \neq 0\).
Hence there exists \(u \in U\) with \(h(u) \in L\), \(h(u) \neq 0\),
so \(u \in U \cap h^{-1}(L)\) and \(u \neq 0\).

Thus \(h^{-1}(L) \, \ess \, K\).

3) Let \(0 \neq X \leq L_1 \cap L_2\).
Then \(X \subseteq L_1\), so \(0 \neq X \cap K_1 \leq L_1\).
But \(X \subseteq L_2\) implies  
\(0 \neq (X \cap K_1) \cap L_2 = X \cap (K_1 \cap K_2)\).
Hence \(K_1 \cap K_2 \, \ess \, L_1 \cap L_2\).

\begin{proposition}
Let \((K_\lambda)_{\lambda \in \Lambda}\),
\((L_\lambda)_{\lambda \in \Lambda}\) be families of submodules of \(M\).
If \((K_\lambda)\) is independent in \(M\) and  
\(K_\lambda \, \ess \, L_\lambda\) for all \(\lambda \in \Lambda\),
then \((L_\lambda)\) is independent in \(M\)
and
\[
\left(\bigoplus_{\lambda \in \Lambda} K_\lambda\right)
   \, \ess \,
\left(\bigoplus_{\lambda \in \Lambda} L_\lambda\right).
\]
\end{proposition}


\textbf{Proof.}

Let \(K_1 \, \ess \, L_1\), \(K_2 \, \ess \, L_2\) with \(K_1 \cap K_2 = 0\).
By 1.8(3), \(0 \, \ess \, L_1 \cap L_2\), hence \(L_1 \cap L_2 = 0\).

Let  
\(\pi_1 : L_1 \oplus L_2 \to L_1\),
\(\pi_2 : L_1 \oplus L_2 \to L_2\)
be the canonical projections.
Since \(K_1 \, \ess \, L_1\) and \(K_2 \, \ess \, L_2\),

\[
\pi_1^{-1}(K_1)
  = K_1 \oplus 0 \, \ess \, L_1 \oplus L_2,
\]
and
\[
\pi_2^{-1}(K_2)
  = 0 \oplus K_2 \, \ess \, L_1 \oplus L_2.
\]

Hence
\[
K_1 \oplus K_2
  = \pi_1^{-1}(K_1) \cap \pi_2^{-1}(K_2)
  \, \ess \, L_1 \oplus L_2.
\]

Induction gives the finite case.
For the general case, let \(0 \neq m \in \oplus_{\lambda \in \Lambda} L_\lambda\).
Then \(m\) lies in a finite direct sum
\(\oplus_{\lambda \in \Lambda_0} L_\lambda\)
for some finite \(\Lambda_0 \subseteq \Lambda\).
Since
\((\oplus_{\lambda \in \Lambda_0} K_\lambda)
   \, \ess \,
(\oplus_{\lambda \in \Lambda_0} L_\lambda)\),
there exists \(r \in R\) with
\(rm \in (\oplus_{\lambda \in \Lambda_0} K_\lambda) \setminus \{0\}
\subseteq (\oplus_{\lambda \in \Lambda} K_\lambda)\).
Thus
\[
\left(\bigoplus_{\lambda \in \Lambda} K_\lambda\right)
   \, \ess \,
\left(\bigoplus_{\lambda \in \Lambda} L_\lambda\right).
\]

\begin{proposition}
Let \(N\) be a submodule of \(M\).
Then there exists a submodule \(Q\) with
\(N \subseteq Q \subseteq M\)
such that \(Q\) is a maximal essential extension of \(N\) inside \(M\).
\end{proposition}

\textbf{Proof.}

Let
\[
\mathfrak{S}
= \{ L \leq M \; ;\; N \subseteq L \subseteq M,\; N \, \ess \, L \},
\]
ordered by inclusion.
Clearly \(\mathfrak{S} \neq \varnothing\), since \(N \in \mathfrak{S}\).

Let \((L_\lambda)_{\lambda \in \Lambda}\) be a totally ordered family of elements of
\(\mathfrak{S}\) and put
\[
L := \bigcup_{\lambda \in \Lambda} L_\lambda.
\]
Clearly \(L \leq M_R\).

Let \(x \in L \setminus \{0\}\).
Then there exists \(\lambda_0 \in \Lambda\) with \(x \in L_{\lambda_0}\).
Since \(N\) is essential in \(L_{\lambda_0}\), there exists \(r \in R\) such that
\(xr \in N\) and \(xr \neq 0\), hence \(L\) is an essential extension of \(N\).
Thus \(\mathfrak{S}\) is inductive and, by Zorn's lemma, \(\mathfrak{S}\) admits a
maximal element \(Q\) which satisfies the required conditions.


\begin{definition}
Let \(M\) be a right \(R\)-module and \(N \leq M_R\).
A submodule \(K \leq M_R\) is called a \emph{complement} of \(N\) in \(M\) if
\(K\) is a maximal submodule of \(M\) with the property that \(K \cap N = 0\).
A submodule \(K \leq M_R\) is called a \emph{complement submodule} of \(M\)
if there exists \(N \leq M_R\) such that \(K\) is a complement of \(N\) in \(M\).
\end{definition}

\begin{remark}
The set
\[
\widetilde{\mathfrak{S}}
  = \{\,L \leq M_R \mid N \cap L = 0\,\}
\]
is inductive and, by applying Zorn's lemma, it follows that there exists a
complement of \(N\) in \(M\).
In particular, \(0\) and \(M\) are complement submodules of \(M\).
\end{remark}


\begin{proposition}
Let \(M_R\), \(N \leq M_R\) and \(K \leq M_R\), with \(K\) a complement of \(N\) in \(M\).
There exists a complement \(Q\) of \(K\) in \(M\) such that \(N \subseteq Q\).
Moreover, \(Q\) is a maximal essential extension of \(N\) in \(M\).
\end{proposition}

\begin{proof}
It is easy to see that the set
\[
\widetilde{\mathfrak{S}}
  = \{\,L \leq M_R \mid K \cap L = 0,\; N \subseteq L\,\}
\]
is inductive, and Zorn's lemma guarantees the existence of \(Q\).

Let \(L\) be a non-zero submodule of \(Q\) such that \(L \cap N = 0\).
Put \(K_1 = L + K\).
Clearly \(K \subseteq K_1\).
If \(x \in N \cap (L + K)\), then \(x = y + z\) with \(y \in L\), \(z \in K\).
But \(z = x - y \in Q\).
Since \(Q \cap K = 0\), we obtain \(z = 0\) and hence \(x = y\).
From \(L \cap N = 0\) it follows that \(x = y = 0\), and therefore
\(N \cap (L + K) = 0\), which contradicts the fact that \(K\) is a complement of
\(N\) in \(M\).
Thus \(L \cap N \neq 0\) for every \(0 \neq L \leq Q\), so \(Q\) is an essential
extension of \(N\).

Suppose that there exists \(Q' \leq M_R\) with \(N \ess Q'\) and \(Q \subsetneq Q'\).
Since \(Q'\) is a complement of \(K\), we have \(Q' \cap K \neq 0\).
But \(N \cap (Q' \cap K) = 0\) and
\(0 \neq Q' \cap K \leq Q'\), contradicting \(N \ess Q'\).
Hence \(Q\) is a maximal essential extension of \(N\) in \(M\).
\end{proof}


\begin{definition}
A submodule \(N\) of \(M_R\) is called \emph{closed} if \(N\) has no proper
(meaning different from \(N\)) essential extension in \(M\).
\end{definition}


\begin{corollary}
Let \(M_R\) be a right \(R\)-module.
The complement submodules of \(M\) coincide with the closed submodules of \(M\).
\end{corollary}

\begin{proof}
From 1.13 it follows immediately that every closed submodule of \(M\) is a
complement submodule of \(M\).

Conversely, let \(K\) be a complement submodule of \(M_R\).
Then there exists \(N \leq M_R\) such that \(K\) is a complement of \(N\) in \(M\).
Assume that \(K\) has a proper essential extension in \(M\); that is, there exists
\(K' \leq M_R\) with \(K \ess K'\) and \(K \subsetneq K'\).
By the maximality of \(K\) we have \(K' \cap N \neq 0\), and since \(K \ess K'\),
it follows that
\[
K \cap K' \cap N \neq 0,
\]
contradiction.
\end{proof}

\begin{corollary}
Let \(N\) be a submodule of \(M_R\).
If \(K\) is a complement of \(N\) in \(M\), then:
\begin{enumerate}
  \item \((N + K) \ess M_R\).
  \item The canonical morphism \(\pi_K \circ i_N : N \to M/K\) is an essential monomorphism.
\end{enumerate}
\end{corollary}

\begin{proof}
(1) Let \(x \in M \setminus \{0\}\).
If \(x \notin K\), then \(K + Rx \neq K\) and hence
\(N \cap (K + Rx) \neq 0\).
Let \(y \in N \cap (K + Rx)\), \(y \neq 0\).
There exist \(z \in K\) and \(r \in R\) such that \(y = z + rx\).
If \(rx = 0\), then \(y = z\) and, since \(N \cap K = 0\), we obtain \(y = 0\),
a contradiction.
Thus \(rx \neq 0\) and, because \(rx = y - z\), we have \(rx \in N + K\),
which shows that \((N + K) \ess M_R\).

(2) We have \(\mathrm{Im}(\pi_K \circ i_N) = (N + K)/K\).
Let \(L/K\) be a non-zero submodule of \(M/K\).
Then
\[
\frac{N + K}{K} \cap \frac{L}{K}
   = \frac{(N + K) \cap L}{K}
   = \frac{N \cap L + K}{K}.
\]
Since \(K\) is a complement of \(N\), we have \(N \cap L \neq 0\), and hence
\[
\frac{N \cap L + K}{K} \neq 0,
\]
which shows that \(\pi_K \circ i_N\) is an essential monomorphism.
\end{proof}



\chapter{Injective Modules}
\section{Injective Module}
Let \(Q\) and \(M\) be two right \(R\)-modules.
We say that \(Q\) is \emph{\(M\)-injective} if for every monomorphism
\(u : M' \to M\) and every morphism \(f : M' \to Q\), there exists
\(g : M \to Q\) such that \(g \circ u = f\); that is, the diagram

\[
\begin{tikzcd}
0 \arrow[r] & M' \arrow[r,"u"] \arrow[d,"f"'] & M \arrow[dl,dashed,"g"] \\
            & Q &
\end{tikzcd}
\]

is commutative.

This property is equivalent to the condition that the map
\[
\Hom(u,Q) : \Hom(M,Q) \longrightarrow \Hom(M',Q)
\]
is surjective for every monomorphism \(u : M' \to M\).
Since the functor \(\Hom(-,Q)\) is left exact, it follows that \(Q\) is
\(M\)-injective if and only if \(\Hom(-,Q)\) is exact with respect to every
short exact sequence of the form
\[
0 \longrightarrow M' \longrightarrow M \longrightarrow M'' \longrightarrow 0.
\]

The \(R\)-module \(Q\) is called \emph{quasi-injective} (or
\emph{self-injective}) if it is \(Q\)-injective.
If \(Q\) is \(M\)-injective for every \(R\)-module \(M\), then \(Q\) is called
\emph{injective}.


2.1.1 Proposition

\begin{proposition}
Let \(Q\) and \(M\) be two \(R\)-modules.
The following statements are equivalent:
\begin{enumerate}
  \item \(Q\) is \(M\)-injective.
  \item For every submodule \(N\) of \(M\) and every morphism
        \(f : N \to Q\), there exists \(g : M \to Q\) such that
        \(g_{\mid N} = f\).
  \item For every essential submodule \(N\) of \(M\) and every morphism
        \(f : N \to Q\), there exists \(g : M \to Q\) such that
        \(g_{\mid N} = f\).
\end{enumerate}
\end{proposition}

\begin{proof}
Implications \((1) \Rightarrow (2)\) and \((2) \Rightarrow (3)\) are obvious.

\medskip
\noindent\((2) \Rightarrow (1)\).
Let \(M'_R\), \(0 \longrightarrow M' \xrightarrow{u} M\) and
\(f : M' \to Q\).
Then \(u(M') \leq M\).
Consider \(i : u(M') \to M\) the canonical injection and
\(\bar{u} : M' \to u(M')\) the isomorphism induced by \(u\).
There exists \(g : M \to Q\) such that \(g \circ i = f \circ \bar{u}^{-1}\).
Hence
\[
g \circ i \circ \bar{u} = f
\quad\text{and therefore}\quad
g \circ u = f.
\]
\[
\begin{tikzcd}
0 \arrow[r] &
M' \arrow[r,shift left=0.35ex,"\bar{u}"]
   \arrow[d,"f"'] &
u(M') \arrow[r,hook,"i"] &
M \arrow[dll,dashed,"g"] \\
& Q &
\end{tikzcd}
\]


\noindent\((3) \Rightarrow (2)\).
Let \(N\) be a submodule of \(M\) and \(K\) a complement of \(N\) in \(M\).
Then \((N \oplus K) \ess M\).
Define \(h : N \oplus K \to Q\) by \(h(n+k) = f(n)\) for all
\(n \in N\), \(k \in K\).
Since \(N \cap K = 0\), the map \(h\) is well-defined.
There exists \(g : M \to Q\) such that \(g_{\mid N \oplus K} = h\), and hence
\(g_{\mid N} = h_{\mid N} = f\).

\[
\begin{tikzcd}
0 \arrow[r] &
N \arrow[r,hook] \arrow[d,"f"'] &
N \oplus K \arrow[r,hook] \arrow[dl,bend right=5,"h"] &
M \arrow[dll,dashed,bend left=10,"g"] \\
& Q &
\end{tikzcd}
\]
\end{proof}


\begin{proposition}
Let \((M_\alpha)_{\alpha \in \Lambda}\) be a family of \(R\)-modules and
\(M\) an \(R\)-module.
Then \(\displaystyle \prod_{\alpha \in \Lambda} M_\alpha\) is
\(M\)-injective if and only if each \(M_\alpha\) is \(M\)-injective.
\end{proposition}

\begin{proof}
Let \(N\) be a submodule of \(M\).
Put \(P = \displaystyle\prod_{\alpha \in \Lambda} M_\alpha\) and let
\(\pi_\alpha : P \to M_\alpha\) be the canonical projections for all
\(\alpha \in \Lambda\).

\medskip
\noindent\("\Leftarrow"\)
Given a morphism \(f : N \to P\), the morphisms
\(\pi_\alpha \circ f : N \to M_\alpha\) can be extended to
\(g_\alpha : M \to M_\alpha\).
There exists \(g : M \to P\) such that \(g_{\mid N} = f\).

\[
\begin{tikzcd}
0 \arrow[r] &
N \arrow[r,hook] \arrow[d,"f"'] &
M \arrow[dl,bend left=5,"g"'] \arrow[ddl,bend left=5,"g_\alpha"] \\
& P \arrow[d,"\pi_\alpha"'] & \\
& M_\alpha &
\end{tikzcd}
\]


\noindent\("\Rightarrow"\) Let \(\forall \alpha \in \Lambda\) and \(f : N \to M_\alpha\).
Considering the canonical inclusion
\(\varepsilon_\alpha : M_\alpha \to P\),
since \(P\) is \(M\)-injective, there exists \(g : M \to P\) which extends
\(\varepsilon_\alpha \circ f : N \to P\).
Then \(\varepsilon_\alpha : M_\alpha \to P\) extends \(f\) and hence
\(M_\alpha\) is \(M\)-injective.

\[
\begin{tikzcd}
0 \arrow[r] &
N \arrow[r, hook] \arrow[d,"f"'] &
M \arrow[ddl,bend left=1,"g"'] \arrow[dddl,bend left=9,"\pi_\alpha g"] \\
& M_\alpha \arrow[d,"\varepsilon_\alpha"'] & \\
& P \arrow[d,"\pi_\alpha"'] & \\
& M_\alpha &
\end{tikzcd}
\]

\end{proof}

\begin{corollary}
\leavevmode
\begin{enumerate}
  \item Let \((Q_\alpha)_{\alpha \in \Lambda}\) be a family of \(R\)-modules.
        Then \(\prod_{\alpha \in \Lambda} Q_\alpha\) is injective
        if and only if each \(Q_\alpha\) is injective for every \(\alpha\in\Lambda\).
  \item The module \(Q_1 \oplus Q_2\) is an injective \(R\)-module if and only if
        each \(Q_i\) is injective for \(i=1,2\).
        In particular, a direct summand of an injective module is injective.
\end{enumerate}
\end{corollary}

\begin{proposition}
Let \(Q\) be an \(R\)-module.
\begin{enumerate}
  \item If \(0 \to M' \xrightarrow{f} M \xrightarrow{g} M'' \to 0\) is
        an exact sequence of \(R\)-modules and \(Q\) is \(M\)-injective,
        then \(Q\) is \(M'\)-injective and \(M''\)-injective.
  \item If \((M_\alpha)_{\alpha \in \Lambda}\) is a family of submodules of \(M\)
        such that \(M = \sum_{\alpha\in\Lambda} M_\alpha\) and
        \(Q\) is \(M_\alpha\)-injective for every \(\alpha\), then \(Q\) is \(M\)-injective.
  \item Let \((N_\alpha)_{\alpha\in\Lambda}\) be a family of \(R\)-modules.
        Then \(Q\) is \(\bigoplus_{\alpha\in\Lambda} N_\alpha\)-injective
        if and only if \(Q\) is \(N_\alpha\)-injective for every \(\alpha\in\Lambda\).
\end{enumerate}
\end{proposition}

\begin{proof}
To show that \(Q\) is \(M'\)-injective, we consider \(N\) a submodule of
\(M'\) and \(\varphi : N \rightarrow Q\) a morphism of \(R\)-modules.
Since \(Q\) is \(M\)-injective, there exists \(\phi : M \to Q\) such that
\(g \circ f\!\mid_{N} = \varphi\), and hence \(\psi \circ f : M' \to Q\) is a morphism
which extends \(\varphi\).

\[
\begin{tikzcd}
0 \arrow[r] &
N \arrow[r] \arrow[d,"\varphi"'] &
M' \arrow[r,"f"] \arrow[dl] &
M \arrow[dll,bend left=5,"\psi"'] \\
& Q & &
\end{tikzcd}
\]

Let \(h : L \to M''\) be a monomorphism. We may assume, without loss of generality,
that \(M' \le M\) and \(M'' = M / M'\). Since \(L \cong h(L) \le M''\), there exist
\(P \le M\), \(M' \subseteq P\) such that \(h(L) = P / M'\) and hence \(L \cong P/M'\).
We obtain the commutative diagram:

\[
\begin{tikzcd}
& & 0 \arrow[d] & 0 \arrow[d] & \\
0 \arrow[r] &
M' \arrow[r] \arrow[d, equal] &
P \arrow[r] \arrow[d] &
L \arrow[r] \arrow[d,"h"] &
0 \\
0 \arrow[r] &
M' \arrow[r] &
M \arrow[r] &
M'' \arrow[r] &
0
\end{tikzcd}
\]

Since \(Q\) is \(M\)-injective, applying the functor \(\Hom(-,Q)\) we obtain
the commutative diagram:

\[
\begin{tikzcd}
0 \arrow[r] &
\Hom(M'',Q) \arrow[r] \arrow[d,"h^{*}"'] &
\Hom(M,Q) \arrow[r] \arrow[d] &
\Hom(M',Q) \arrow[r] \arrow[d,equal] &
0 \\
0 \arrow[r] &
\Hom(L,Q) \arrow[r] &
\Hom(P,Q) \arrow[r] \arrow[d] &
\Hom(M',Q) &
\\
& & 0 & &
\end{tikzcd}
\]

We obtain that \(h^{*} = \Hom(h,Q)\) is an epimorphism, which shows that
\(Q\) is \(M''\)-injective.

\medskip

2) Let \(N\) be a submodule of \(M\) and \(f : N \to Q\) a morphism of \(R\)-modules.
Consider the set
\[
\mathfrak{S} = \{(L,h) \mid N \le L \le M,\; h : L \to Q,\; h\!\mid_{N} = f\}.
\]
Since \((N,f) \in \mathfrak{S}\), we have \(\mathfrak{S} \ne \varnothing\).
Define on \(\mathfrak{S}\) the order relation
\((L_1,h_1) \preccurlyeq (L_2,h_2)\) if and only if
\(L_1 \le L_2\) and \(h_2\!\mid_{L_1} = h_1\).
One checks that \(\mathfrak{S}\) is inductive and, by Zorn’s lemma,
there exists a maximal element \((L_0,g_0)\) of \(\mathfrak{S}\).
To show that \(L_0 = M\) it is enough to prove that
\(M_\alpha \le L_0\) for every \(\alpha \in \Lambda\).

Consider the diagram
\[
\begin{tikzcd}
0 \arrow[r] &
L_0 \cap M_\alpha \arrow[r,"i_\alpha"] \arrow[d,"i_0"'] &
M_\alpha \arrow[ddl,"h_\alpha"] \\
& L_0 \arrow[d,"g_0"'] & \\
& Q &
\end{tikzcd}
\]

from which it follows that there exists \(h_\alpha : M_\alpha \to Q\) such that
\(h_\alpha \circ i_\alpha = g_0 \circ i_\alpha\).
Define
\(h^{*} : L_0 + M_\alpha \to Q\) by
\(h^{*}(l + m_\alpha) = g_0(l) + h_\alpha(m_\alpha)\),
for all \(l \in L_0\), \(m_\alpha \in M_\alpha\).
If \(l + m_\alpha = 0\), then \(l = -m_\alpha \in L_0 \cap M_\alpha\) and hence
\(h^{*}(l + m_\alpha) = g_0(l) + h_\alpha(l)\), which shows that \(h^{*}\) is well defined.
Thus \((L_0 + M_\alpha, h^{*}) \in \mathfrak{S}\), and since
\((L_0,g_0) \preccurlyeq (L_0 + M_\alpha, h^{*})\),
by the maximality of \((L_0,g_0)\) we obtain
\(L_0 = L_0 + M_\alpha\), i.e. \(M_\alpha \le L_0\) for every \(\alpha \in \Lambda\).

3) “\(\Rightarrow\)” Since \(N_\alpha \le N\) and \(Q\) is \(N\)-injective,
it follows that \(Q\) is \(N_\alpha\)-injective for every \(\alpha \in \Lambda\).

“\(\Leftarrow\)” Let \(N'_\alpha = i_\alpha(N_\alpha)\).
Since \(Q\) is \(N_\alpha\)-injective and \(N'_\alpha \cong N_\alpha\),
we see that \(Q\) is \(N'_\alpha\)-injective.
Now apply (2).
\end{proof}

\begin{corollary}
\leavevmode
\begin{enumerate}
  \item The module \(Q_1 \oplus Q_2\) is a quasi-injective \(R\)-module
        if and only if each \(Q_i\) is \(Q_j\)-injective for all \(i,j = 1,2\).
        In particular, a direct summand of a quasi-injective module is quasi-injective.
  \item The module \(Q^{n}\) is quasi-injective over \(R\) if and only if
        \(Q\) is quasi-injective.
\end{enumerate}
\end{corollary}

\begin{corollary}
Let \(Q\) and \(M\) be two \(R\)-modules.
Then \(Q\) is \(M\)-injective if and only if \(Q\) is \(mR\)-injective
for every \(m \in M\).
\end{corollary}

\begin{proof}
The implication “\(\Rightarrow\)” is clear.

For “\(\Leftarrow\)”, since \(M = \sum_{m \in M} mR\), it follows from 2.1.4(2)
that \(Q\) is \(M\)-injective.
\end{proof}

\begin{theorem}[Baer’s criterion]\label{thm:Baer-en}
For an \(R\)-module \(Q\) the following statements are equivalent:
\begin{enumerate}
  \item \(Q\) is injective.
  \item \(Q\) is \(R\)-injective.
  \item For every right ideal \(I\) of \(R\) and every morphism \(f : I \to Q\)
        there exists \(x \in Q\) such that \(f(a) = xa\) for all \(a \in I\).
\end{enumerate}
\end{theorem}

\begin{proof}
The implication \((1) \Rightarrow (2)\) is obvious.

\((2) \Rightarrow (1)\).
Let \(M\) be an \(R\)-module and \(x \in M\).
Since \(\varphi_x : R \to xR\), \(\varphi_x(a) = xa\) for all \(a \in R\),
is a surjective morphism of \(R\)-modules, it follows that
\(R / \Ker \varphi_x \cong xR\).
As \(Q\) is \(R\)-injective, 2.1.4(1) implies that
\(Q\) is \(R / \Ker \varphi_x\)-injective, and hence
\(Q\) is \(xR\)-injective for every \(x \in M\).
Therefore, by 2.1.6 we obtain that \(Q\) is \(M\)-injective.

\((2) \Rightarrow (3)\).
Let \(I\) be a right ideal of \(R\) and \(f : I \to Q\).
There exists \(g : R \to Q\) such that \(g\!\mid_{I} = f\).
Put \(x = g(1) \in Q\). Then
\(f(a) = g(a) = ag(1) = xa\) for all \(a \in I\).

\((3) \Rightarrow (2)\).
Suppose that for a morphism \(f : I \to Q\) there exists
\(x \in Q\) with \(f(a) = xa\) for all \(a \in I\).
Define \(g : R \to Q\) by \(g(r) = xr\) for all \(r \in R\).
Then clearly \(g\!\mid_{I} = f\).
\end{proof}

\begin{definition}
\leavevmode
\begin{enumerate}
  \item An \(R\)-module \(Q\) is called \emph{divisible} if for every \(y \in Q\)
        and every \(a \in R\) which is not a zero divisor, there exists \(x \in Q\)
        such that \(ax = y\).
        It is easily seen that any factor module of a divisible module is divisible.
   \item A commutative integral domain is called a \emph{PID ring}
        (principal ideal domain) if every ideal of it is principal.
\end{enumerate}
\end{definition}

\begin{proposition}
\leavevmode
\begin{enumerate}
  \item Every injective module is divisible.
  \item Let \(R\) be a PID.
    \begin{enumerate}
      \item[(i)] If \(Q\) is an \(R\)-module, then \(Q\) is injective
                 if and only if it is divisible.
      \item[(ii)] If \(I\) is a non-zero ideal of \(R\), then \(R/I\) is
                  a quasi-injective \(R\)-module. In particular,
                  \(\mathbb{Z}_n\) is a quasi-injective \(\mathbb{Z}\)-module
                  for every \(n \ge 1\).
    \end{enumerate}
\end{enumerate}
\end{proposition}

\begin{proof}
\leavevmode

\noindent 1) Let \(Q\) be a divisible \(R\)-module, \(y \in Q\) and
\(a \in R\) a non–zero divisor. Define
\(f : aR \to Q\) by \(f(ax) = yx\) for all \(x \in R\).
Since \(a\) is not a zero divisor, \(f\) is well defined.
By Baer’s criterion there exists \(x \in Q\) such that
\(f(\lambda) = x\lambda\) for all \(\lambda \in aR\).
In particular \(y = f(a) = f(a\cdot 1) = xa\); hence \(Q\) is divisible.

\medskip

\noindent 2)(i) The implication “\(\Rightarrow\)” follows immediately from (1).

\smallskip

\noindent “\(\Leftarrow\)” Let \(Q\) be a divisible \(R\)-module and
\(I\) a right ideal of \(R\).
Then there exists \(a \in R\) with \(I = aR\).
Given a morphism of \(R\)-modules \(f : I \to Q\),
choose \(x \in Q\) with \(f(a) = xa\). For any \(r \in R\),
\(f(ar) = f(a)r = xar\), so by Baer’s criterion \(Q\) is injective.

\medskip

\noindent 2)(ii) Let \(I = aR\) and \(J = bR\) be non-zero ideals of \(R\)
with \(I \subseteq J\), and let
\(f : J/I \to R/I\) be a morphism of \(R\)-modules.
There exists \(c \in R\) such that \(a = bc\).
Write \(f(\widehat{b}) = \widehat{x} \in R/I\).
Then \(xc \in I = aR\), so there is \(a_1 \in R\) with \(xc = aa_1\),
and hence \(x = ba_1\).

Define \(g : R/I \to R/I\) by
\(g(\widehat{r}) = \widehat{a_1 r}\) for all \(r \in R\).
Then \(g\) is an \(R\)-module homomorphism and
\(g(\widehat{b}) = \widehat{x}\), so \(g\!\mid_{J/I} = f\).
Thus \(R/I\) is quasi-injective.  
In particular, for \(R = \mathbb{Z}\) and \(I = n\mathbb{Z}\) we obtain that
\(\mathbb{Z}_n\) is a quasi-injective \(\mathbb{Z}\)-module for every \(n \ge 1\).
\end{proof}

\begin{corollary}
An abelian group \(G\) is an injective \(\mathbb{Z}\)-module
if and only if \(G\) is divisible.
\end{corollary}



\begin{corollary}
\leavevmode
\begin{enumerate}
  \item \(\mathbb{Q}\) and \(\mathbb{Z}_{p^\infty}\) are injective \(\mathbb{Z}\)-modules.
  \item Any direct sum of injective \(\mathbb{Z}\)-modules is an injective
        \(\mathbb{Z}\)-module.
  \item Any factor group of an injective \(\mathbb{Z}\)-module is injective.
\end{enumerate}
\end{corollary}



\begin{lemma}
Let \(A,S,T\) be rings and let \( {}_{A}M_{S}, {}_{A}N_{T}\) be bimodules.
Then \(\Hom_A(M,N)\) has a structure of left \(S\)-module and right \(T\)-module
given by
\[
  (s \cdot f)(x) = f(xs), \qquad (f \cdot t)(x) = f(x)t,
\]
for \(s \in S\), \(t \in T\), \(x \in M\), \(f \in \Hom_A(M,N)\).
\end{lemma}

\begin{proof}
Let \(a,b \in A\) and \(x,y \in M\).
Then
\[
 (s \cdot f)(ax+by) = f((ax+by)s)
   = f((ax)s) + f((by)s)
   = af(xs) + bf(ys)
   = a(s \cdot f)(x) + b(s \cdot f)(y),
\]
so \(s \cdot f \in \Hom_A(M,N)\). Similarly \(f \cdot t \in \Hom_A(M,N)\).

For \(s,s' \in S\) and \(f,g \in \Hom_A(M,N)\) we have
\begin{align}
 (s \cdot (f+g))(x)
   &= (f+g)(xs)
    = f(xs)+g(xs)
    = (s \cdot f)(x)+(s \cdot g)(x),
    \tag{1} \\
 ((s+s') \cdot f)(x)
   &= f(x(s+s'))
    = f(xs+xs')
    = f(xs)+f(xs')
    = (s \cdot f)(x)+(s' \cdot f)(x),
    \tag{2} \\
 ((ss') \cdot f)(x)
   &= f(xss')
    = f((xs)s')
    = (s' \cdot f)(xs)
    = (s \cdot (s' \cdot f))(x),
    \tag{3} \\
 (1_S \cdot f)(x)
   &= f(x1_S) = f(x).
    \tag{4}
\end{align}
From (1)–(4) we see that \(\Hom_A(M,N)\) is a left \(S\)-module.
A similar computation shows that \(\Hom_A(M,N)\) is a right \(T\)-module.
Moreover,
\(((s \cdot f) \cdot t)(x) = (s \cdot f)(xt) = f(xts)t\)
and
\((s \cdot (f \cdot t))(x) = f(xts)t\),
so the two actions commute and \(\Hom_A(M,N)\) is an \(S\)–\(T\) bimodule.
\end{proof}


\begin{proposition}[Eckmann–Schopf]
Let \(Q\) be a divisible abelian group. Then the left \(R\)-module
\(\Hom_{\mathbb{Z}}(R,Q)\) is injective.
\end{proposition}

\begin{proof}
By the previous lemma, $\Hom_{\mathbb{Z}}(R,Q)$ carries a structure
of left $R$–module given by
\[
(r \cdot f)(a) = f(ar), \qquad \forall a,r \in R,\ f \in \Hom_{\mathbb{Z}}(R,Q).
\]
Let $I$ be a left ideal of $R$ and
$h : I \to \Hom_{\mathbb{Z}}(R,Q)$ a morphism of left $R$–modules.
Define
\[
\gamma : \mathbb{Z}I \longrightarrow \mathbb{Z}Q, \qquad
\gamma(a) = h(a)(1).
\]
Then $\gamma$ is a morphism of $\mathbb{Z}$–modules.
Since $Q$ is $\mathbb{Z}$–injective, there exists
$\widetilde{\gamma} : \mathbb{Z}R \to \mathbb{Z}Q$ such that
$\widetilde{\gamma}_{\mid I} = \gamma$.
For $a \in I$ and $r \in R$ we have
\[
(a \cdot \widetilde{\gamma})(r)
  = \widetilde{\gamma}(ra)
  = h(ra)(1)
  = (r \cdot h(a))(1)
  = h(a)(r),
\]
hence $h(a) = a \cdot \widetilde{\gamma}$ for all $a \in I$.
By Baer’s criterion, $\Hom_{\mathbb{Z}}(R,Q)$ is an injective
left $R$–module.
\end{proof}


\begin{proposition}
Every left $R$–module $M$ can be embedded into an injective
left $R$–module.
\end{proposition}

\begin{proof}
There exists a free abelian group $\mathbb{Z}^{(A)}$ and a surjective
$\mathbb{Z}$–morphism $f : \mathbb{Z}^{(A)} \to M$.
Hence
\[
\mathbb{Z}M \cong \mathbb{Z}^{(A)}/\ker f
  \subseteq \mathbb{Q}^{(A)}/\ker f,
\]
and therefore there is a divisible abelian group $G$ with
$\mathbb{Z}M \subseteq \mathbb{Z}G$.
Applying the functor $\Hom_{\mathbb{Z}}(R,-)$ we obtain a monomorphism
\[
{}_R M \cong \Hom_{\mathbb{Z}}(R,M)
   \hookrightarrow \Hom_{\mathbb{Z}}(R,G).
\]
Since $G$ is divisible, Proposition~2.1.13 implies that
$\Hom_{\mathbb{Z}}(R,G)$ is an injective left $R$–module.
Thus $M$ embeds into an injective left $R$–module.
\end{proof}


\begin{proposition}
Let $Q$ be an $R$–module. Then $Q$ is injective if and only if every
short exact sequence
\[
0 \longrightarrow Q \xrightarrow{f} M \xrightarrow{g} M' \longrightarrow 0
\]
splits.
\end{proposition}

\begin{proof}
“$\Rightarrow$”.
Assume $Q$ is injective.
Exactness implies that $f$ is a monomorphism.
By injectivity of $Q$ there exists $h : M \to Q$ with
$h f = \mathrm{id}_Q$, hence the sequence splits.

“$\Leftarrow$”.
By Proposition~2.1.14 there is an injective $R$–module $Q'$ and a
monomorphism $i : Q \to Q'$.
Consider the exact sequence
\[
0 \longrightarrow Q \xrightarrow{i} Q' \longrightarrow Q'/i(Q) \longrightarrow 0.
\]
By the hypothesis this sequence splits, so $Q$ is a direct summand of $Q'$.
A direct summand of an injective module is injective, therefore $Q$ is injective.
\end{proof}

\section{Injective envelopes}

\begin{definition}
Let $M$ be an $R$–module.
A pair $(E,i)$ is called an \emph{injective envelope} of $M$ if
$E$ is injective and $i : M \to E$ is an essential monomorphism.
\end{definition}

\begin{proposition}
Let $Q$ be an injective $R$–module.
Then every complement submodule of $Q$ is a direct summand of $Q$.
\end{proposition}

\begin{proof}
Let $K$ be a submodule of $Q$ and $N$ a complement of $K$ in $Q$,
that is, $K \cap N = 0$ and $K+N$ is an essential submodule of $Q$.
Then $(K+N)/N \cong Q/N$.
Define $g : (K+N)/N \to Q$ by
\[
g((x+y)+N) = x, \qquad x \in K,\ y \in N.
\]
Since $K \cap N = 0$, $g$ is well defined and injective.
As $Q$ is injective, there exists $h : Q/N \to Q$ with
$h_{\mid (K+N)/N} = g$.
Because $(K+N)/N \cong Q/N$ and $g$ is a monomorphism, $h$ is also a
monomorphism.
We have
$K = \mathrm{Im}\,g = h((K+N)/N) \subseteq h(Q/N)$.
As $K$ is a closed submodule, it follows that $K = h(Q/N)$.
Since $h$ is a monomorphism, $(K+N)/N = Q/N$, hence $K+N = Q$.
Thus $K$ is a direct summand of $Q$.
\end{proof}

\begin{theorem}[Eckmann--Schopf]
Every $R$–module $M$ has an injective envelope, unique up to isomorphism.
\end{theorem}

\begin{proof}
By Proposition~2.1.14 there exists an injective $R$–module $Q$ with
$M \le Q$.
Let $E$ be a maximal essential extension of $M$ in $Q$.
Then $E$ is a complement submodule of $Q$, and by the previous
proposition $E$ is injective.
Hence $(E,i)$, where $i : M \hookrightarrow E$ is the inclusion,
is an injective envelope of $M$.

For uniqueness, let $(E_1,i_1)$ and $(E_2,i_2)$ be two injective
envelopes of $M$.
Since $E_2$ is injective, there exists $f : E_1 \to E_2$ with
$f i_1 = i_2$.
The map $i_2$ is a monomorphism and $i_1$ is an essential monomorphism,
so (using 1.4) $f$ is a monomorphism.
Thus $E_1 \cong f(E_1)$ and $E_2 = f(E_1) \oplus E_3$ for some submodule
$E_3$.
But $i_2(M) \subseteq f(E_1)$, hence $i_2(M) \cap E_3 = 0$.
Since $i_2$ is essential, we must have $E_3 = 0$, so $E_2 = f(E_1)$ and
$f$ is an isomorphism.
\end{proof}

In practice we fix one representative of this isomorphism class and
denote it by $E(M)$, with $M \ess E(M)$.

\begin{proposition}
Let $M$ be an $R$–module and $i : M \to Q$ a monomorphism with $Q_R$
injective.
The following statements are equivalent:
\begin{enumerate}
  \item $(Q,i)$ is an injective envelope of $M$;
  \item for every monomorphism $f : M \to Q'$ with $Q'$ injective,
        there exists a monomorphism $g : Q \to Q'$ such that $g i = f$.
\end{enumerate}
\end{proposition}

\begin{proof}
$(1) \Rightarrow (2)$.
Let $f : M \to Q'$ be a monomorphism with $Q'$ injective.
By injectivity of $Q'$ there is $u : Q \to Q'$ with $u i = f$.
Because $i$ is an essential monomorphism and $Q'$ is injective,
the image $u(Q)$ is a complement of $f(M)$; by the definition of
injective envelope this forces $u$ to be a monomorphism.
Set $g = u$.

$(2) \Rightarrow (1)$.
Let $(E(M),j)$ be an injective envelope of $M$.
Applying (2) to $f=j$ we obtain a monomorphism
$g : Q \to E(M)$ with $g i = j$.
Since $j$ is an essential monomorphism, it follows that $i$ is also
essential; hence $(Q,i)$ is an injective envelope of $M$.
\end{proof}


\begin{proposition}
For any family of right $R$–modules $M_1,M_2,\dots,M_n$ we have
\[
E\!\left(\bigoplus_{i=1}^{n} M_i\right)
  \cong \bigoplus_{i=1}^{n} E(M_i).
\]
\end{proposition}

\begin{proof}
By 1.9, $\bigoplus_{i=1}^{n} E(M_i)$ is an essential extension of
$\bigoplus_{i=1}^{n} M_i$.
Moreover,
\[
\bigoplus_{i=1}^{n} E(M_i) \cong \prod_{i=1}^{n} E(M_i),
\]
so by 2.1.3 the module $\bigoplus_{i=1}^{n} E(M_i)$ is injective.
By uniqueness of injective envelopes we obtain
\[
E\!\left(\bigoplus_{i=1}^{n} M_i\right)
  \cong \bigoplus_{i=1}^{n} E(M_i).
\]
\end{proof}


% --- English: Theorem 2.2.6 + proof ---

\begin{theorem}
Let \(Q\) and \(M\) be two \(R\)-modules. Then \(Q\) is \(M\)-injective if and only if
\(f(M) \subseteq Q\) for every \(f \in \Hom(E(M),E(Q))\).
\end{theorem}

\begin{proof}
“\(\Rightarrow\)” Let \(f \in \Hom(E(M),E(Q))\) and set
\[
K := \{\,m \in M \mid f(m) \in Q\,\}.
\]
Since \(Q\) is \(M\)-injective, there exists a morphism \(\bar{f} : M \to Q\) such that
\(\bar{f}\!\mid_{K} = f\!\mid_{K}\).
We claim that
\[
Q \cap (\bar{f}-f)(M) = 0.
\]
Take \(x \in Q\) and \(m \in M\) with \(x = (\bar{f}-f)(m)\).
Then
\[
f(m) = \bar{f}(m) - x \in Q,
\]
so \(m \in K\).
Hence
\[
x = \bar{f}(m) - f(m) = f(m) - f(m) = 0,
\]
and therefore \(Q \cap (\bar{f}-f)(M) = 0\).
Since \(Q\) is an essential submodule of \(E(Q)\), it follows that
\((\bar{f}-f)(M) = 0\).
Thus \(f(M) = \bar{f}(M) \subseteq Q\).

“\(\Leftarrow\)” As \(E(Q)\) is injective, it is enough to work with
\(f \in \Hom(M,E(Q))\).
Let \(N\) be a submodule of \(M\) and \(g : N \to Q\) a morphism of \(R\)-modules.
Because \(E(Q)\) is injective, there exists \(\tilde{g} : M \to E(Q)\) such that
\(\tilde{g}\!\mid_{N} = i \circ g\), where \(i : Q \to E(Q)\) is the canonical
injection.
By hypothesis, \(\tilde{g}(M) \subseteq Q\), so identifying \(\tilde{g}\) with its
corestriction to \(Q\) we obtain a morphism \(h : M \to Q\) with
\(h\!\mid_{N} = g\).
Therefore \(Q\) is \(M\)-injective.
\end{proof}


% --- English: Corollary 2.2.7 ---

\begin{corollary}
An \(R\)-module \(Q\) is quasi-injective if and only if
\(f(Q) \subseteq Q\) for every \(f \in \End(E(Q))\).
\end{corollary}


% --- English: Theorem 2.2.8 (Matlis–Bass) + proof ---

\begin{theorem}[Matlis--Bass]
Let \(R\) be a ring. Then \(R\) is right noetherian if and only if,
for every simple right \(R\)-module \(S_i\) (\(i \geq 1\)),
\[
Q := \bigoplus_{i=1}^{\infty} E(S_i)
\]
is an injective right \(R\)-module.
\end{theorem}

\begin{proof}
“\(\Rightarrow\)” Let \(L\) be a right ideal of \(R\),
\[
Q = \bigoplus_{i=1}^{\infty} E(S_i)
\]
and let \(f : L \to Q\) be a morphism of right \(R\)-modules.
There exist elements \(a_1,\dots,a_n \in L\) such that
\[
L = a_1 R + a_2 R + \cdots + a_n R.
\]
Clearly, there is \(m \geq 1\) with
\(f(a_k) \in \bigoplus_{j=1}^{m} E(S_j)\) for all \(k = 1,\dots,n\), hence
\[
\Im f \subseteq \bigoplus_{j=1}^{m} E(S_j).
\]
Since \(\bigoplus_{j=1}^{m} E(S_j)\) is injective, there exists
\(g : R \to \bigoplus_{j=1}^{m} E(S_j)\) such that \(g\!\mid_{L} = f\).
Let \(\bar{f} = i \circ g\), where
\(i : \bigoplus_{j=1}^{m} E(S_j) \to Q\) is the canonical injection.
Then \(\bar{f}\!\mid_{L} = f\), so \(Q\) is injective.

“\(\Leftarrow\)” Suppose that \(R\) is not right noetherian.
Then there exists a strictly ascending chain of finitely generated right ideals:
\[
L_1 \subsetneq L_2 \subsetneq \cdots \subsetneq L_n \subsetneq \cdots .
\]
By Krull’s lemma, for every \(n \geq 1\) there exists a maximal submodule
\(M_n \subsetneq L_n\) such that
\[
L_{n-1} \subseteq M_n \quad \text{for all } n \geq 2.
\]
Set
\[
L := \bigcup_{k=1}^{\infty} L_k, \qquad
\pi_k : L_k \longrightarrow L_k/M_k
\]
for the canonical projections, and put
\[
E_k := E(L_k/M_k) \quad \text{for each } k \geq 1.
\]
Then
\[
E := \bigoplus_{k=1}^{\infty} E_k
\]
is injective, and
\[
f : L \longrightarrow E, \qquad
f(a) = \sum_{k=1}^{\infty} \pi_k(a)
\]
is well defined.
There exists an element
\(x \in E_1 \oplus \cdots \oplus E_n\) such that
\(f(a) = x a\) for all \(a \in L\).
It follows that \(\pi_k(a) = 0\) for every \(k \geq n+1\),
that is, \(a \in M_k\) for all \(k \geq n+1\).
Hence
\[
L \subseteq M_{n+1} \subsetneq L_{n+1} \subseteq M_{n+2} \subsetneq L_{n+2}
\subseteq \cdots \subseteq L,
\]
a contradiction.
Therefore \(R\) is right noetherian.
\end{proof}



\chapter{Direct Sums of Uniform (Co-irreducible) Modules}

\begin{proposition}
Let \(M\) be an \(R\)-module and \(E(M)\) its injective envelope. Then the
following statements are equivalent:
\begin{enumerate}
  \item \(E(M)\) is indecomposable.
  \item If \(L\) and \(K\) are non–zero submodules of \(M\), then
        \(L \cap K \neq 0\).
  \item If \(x,y \in M\setminus\{0\}\), then there exist \(a,b \in R\) such that
        \(0 \neq xa = yb\).
  \item \(M\) is an essential extension of each of its non–zero submodules.
\end{enumerate}
\end{proposition}

\begin{proof}
\((2) \Rightarrow (1)\).
Assume that \(E(M) = L' \oplus K'\) with \(L'\) and \(K'\) non–zero submodules
of \(E(M)\).
Since \(M \ess E(M)\), we have \(L = L' \cap M \neq 0\) and
\(K = K' \cap M \neq 0\), hence
\[
L \cap K = (L' \cap K') \cap M = 0,
\]
which contradicts (2).

\smallskip
\((1) \Rightarrow (2)\).
Assume there exist non–zero submodules \(L,K \leq M\) such that
\(L \cap K = 0\).
Then \(E(L) \leq E(M)\) and \(K \cap E(L) = 0\) (otherwise, since
\(L \subseteq E(L)\), we would have \(L \cap K \neq 0\)).
The short exact sequence
\[
0 \longrightarrow E(L) \longrightarrow E(M) \longrightarrow E(M)/E(L)
\longrightarrow 0
\]
splits because \(E(L)\) is injective, hence
\(E(M) \cong E(L) \oplus E(M)/E(L)\).
It follows that
\[
0 = K \cap E(L) = K \cap E(M) = K,
\]
a contradiction.

The equivalences \((2) \Leftrightarrow (3)\) and \((2) \Leftrightarrow (4)\) are
straightforward.
\end{proof}


\begin{definition}
An \(R\)-module \(M\) satisfying any of the equivalent conditions above is
called \emph{uniform} (or \emph{co-irreducible}).
Let \(M\) be an \(R\)-module and \(N\) a proper submodule of \(M\).
If \(M/N\) is uniform, we say that \(N\) is \emph{irreducible in \(M\)}.
Clearly, \(N\) is irreducible in \(M\) if and only if from an equality
\(N = P \cap Q\) with \(P,Q \le M\) it follows that \(N = P\) or \(N = Q\).
\end{definition}

\begin{example*}
\leavevmode
\begin{enumerate}
  \item Every simple module is clearly uniform.
  \item \(\mathbb{Z}\) is a uniform \(\mathbb{Z}\)-module.
  \item By Proposition~3.1, an injective module is indecomposable if and only
        if it is uniform. In particular, if \(M\) is a uniform
        \(R\)-module, then \(E(M)\) is uniform. Hence the
        \(\mathbb{Z}\)-modules \(\mathbb{Q}\) and \(\mathbb{Z}_{p^\infty}\)
        are uniform.
  \item The \(\mathbb{Z}\)-module \(\mathbb{Z}_n\) is uniform if and only if
        \(n\) is a power of a prime. If \(n = p^k\) with \(p \ge 2\) and
        \(k \ge 1\), then
        \(\langle p^m \rangle \subseteq \langle p^i \rangle \cap \langle p^j \rangle\)
        for all \(i,j \in \{1,\dots,k-1\}\), where \(m = \min(i,j)\), so
        \(\mathbb{Z}_{p^k}\) is uniform.
        Conversely, assume that \(\mathbb{Z}_n\) is uniform and write
        \(n = p_1^{\alpha_1}\cdots p_s^{\alpha_s}\) as the factorisation of \(n\)
        into primes. If \(s \ge 2\), then
        \[
          \langle p_1^{\alpha_1}\cdots p_{s-1}^{\alpha_{s-1}} \rangle
          + \mathbb{Z}
          \quad\text{and}\quad
          \langle p_s^{\alpha_s} \rangle + \mathbb{Z}
        \]
        are non–zero submodules whose intersection is zero, a contradiction.
        Hence \(s = 1\) and \(n\) is a prime power.
\end{enumerate}
\end{example*}


\begin{proposition}
Let \(M\) be an \(R\)-module and \(N\) a proper submodule of \(M\).
If \(x \in M \setminus N\), then there exists an irreducible submodule
\(P \le M\) such that \(N \subseteq P\) and \(x \notin P\).
\end{proposition}

\begin{proof}
Set
\[
\mathfrak{S} = \{\,N' \le M \mid N \subseteq N' \text{ and } x \notin N'\,\}.
\]
The set \(\mathfrak{S}\), ordered by inclusion, is inductive, so by Zorn's
Lemma it has a maximal element \(P\).
We show that \(P\) is irreducible in \(M\).
Assume that \(P = U \cap V\) with \(U,V \le M\) and \(P \subsetneq U\),
\(P \subsetneq V\).
Since \(x \notin P\), we have \(x \notin U\) or \(x \notin V\); suppose
\(x \notin U\).
Then \(U \in \mathfrak{S}\) and \(P \subsetneq U\), contradicting the
maximality of \(P\).
Hence \(P\) is irreducible.
\end{proof}

\begin{corollary}
Let \(M\) be an \(R\)-module and \(N\) a proper submodule of \(M\).
Then \(N\) is the intersection of irreducible submodules of \(M\).
\end{corollary}

\begin{proof}
This follows immediately from the previous proposition.
\end{proof}


\begin{proposition}
Let \(M\) be a uniform \(R\)-module. Then:
\begin{enumerate}
  \item Every non–zero submodule of \(M\) is uniform.
  \item Every essential extension of \(M\) is uniform.
\end{enumerate}
\end{proposition}


% Further chapters will be added as we progress.
% Example:
% \chapter{Title of Chapter 1}
% ...

\backmatter

% REFERENCES
% As with the RO version, we can either recreate the bibliography
% manually or from a .bib file.
% \bibliographystyle{plain}
% \bibliography{bibliography}

\end{document}
