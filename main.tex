% main.tex — English translation
\documentclass[12pt,a4paper]{book}

% Encoding and language
\usepackage[utf8]{inputenc}
\usepackage[T1]{fontenc}
\usepackage[english]{babel}

% Maths and layout
% \usepackage{amsmath,amssymb,amsthm}
\usepackage{amsmath,amssymb,amsthm}
\DeclareMathOperator{\End}{End}
\DeclareMathOperator{\soc}{soc}

\usepackage{geometry}
\geometry{margin=3cm}
\usepackage{graphicx}
\usepackage{microtype}
\usepackage[hidelinks]{hyperref}

% ---------- Theorem environments (mirroring the RO version) ----------
\theoremstyle{plain}
% \newtheorem{theorem}{Theorem}[chapter]
\newtheorem{theorem}{Theorem}[section]   % [section] not [chapter]

\newtheorem{lemma}[theorem]{Lemma}
\newtheorem{proposition}[theorem]{Proposition}
\newtheorem{corollary}[theorem]{Corollary}

\theoremstyle{definition}
\newtheorem{definition}[theorem]{Definition}
\newtheorem{example}[theorem]{Example}

\theoremstyle{remark}
\newtheorem{remark}[theorem]{Remark}

% ---------- Title page ----------
\title{The Uniform (Co-Irreducible) Dimension of Rings and Modules}
\author{Supervisor: Prof. Dr Tiberiu Dumitrescu\\[0.5em]
Student: Sorin Bogde}
\date{Original thesis: University of Bucharest, Faculty of Mathematics, 1999\\
English translation: \today}

% \date{Original: 1999 \\ Translation: 2025}

\begin{document}

\frontmatter
\maketitle
\tableofcontents

% \chapter*{Preface}
% A short translator's preface can be added here later.

\mainmatter

\setcounter{chapter}{-1}
\chapter{Generalities}

\section{Semisimple modules}

\begin{definition}
A non-zero \(R\)-module \(S\) is called \emph{simple} if its only submodules are
\(0\) and \(S\).
\end{definition}

\begin{proposition}
Let \(S\) be an \(R\)-module. The following statements are equivalent:
\begin{enumerate}
  \item \(S\) is simple;
  \item for every non-zero element \(x \in S\) we have \(S = xR\);
  \item \(S \cong R/I\), where \(I\) is a maximal right ideal.
\end{enumerate}
\end{proposition}

\begin{lemma}[Schur]
Let \(S\) and \(S'\) be simple \(R\)-modules and let \(f\colon S \to S'\) be a homomorphism
of \(R\)-modules. Then \(f = 0\) or \(f\) is an isomorphism. In particular
\(\End_R(S)\) is a division ring.
\end{lemma}

\begin{definition}
Let \(M\) be an \(R\)-module and let \((S_i)_{i\in I}\) be the family of all simple
submodules of \(M\). If \(M = \sum_{i\in I} S_i\), then \(M\) is called \emph{semisimple}.
\end{definition}

\begin{proposition}
Let \(M\) be a semisimple \(R\)-module and \(N\) a submodule of \(M\).
Then there exists a subset \(J \subseteq I\) such that
\begin{enumerate}
  \item the family \((S_j)_{j\in J}\) is independent;
  \item \(M = N \oplus \bigl(\bigoplus_{j\in J} S_j\bigr)\).
\end{enumerate}
\end{proposition}

\begin{corollary}
With the above notation, for the semisimple module \(M\) there exists \(J \subseteq I\) such
that the family \((S_j)_{j\in J}\) is independent and
\[
  M = \bigoplus_{i\in I} M_i.
\]
\end{corollary}

\begin{corollary}
If \(M\) is a semisimple \(R\)-module and \(N\) a submodule of \(M\), then both \(N\) and
\(M/N\) are semisimple.
\end{corollary}

\begin{corollary}
A direct sum of semisimple modules is a semisimple module.
\end{corollary}

\begin{theorem}
Let \(M\) be an \(R\)-module. The following statements are equivalent:
\begin{enumerate}
  \item \(M\) is semisimple;
  \item \(M\) is isomorphic to a direct sum of simple modules;
  \item every submodule of \(M\) is a direct summand of \(M\);
  \item every short exact sequence
  \[
    0 \longrightarrow M' \xrightarrow{\,f\,} M \xrightarrow{\,g\,} M'' \longrightarrow 0
  \]
  splits.
\end{enumerate}
\end{theorem}

\begin{definition}
The sum of all simple submodules of \(M\) is called the \emph{socle} of \(M\) and is denoted
by \(\soc(M)\). If \(M\) has no simple submodule we put \(\soc(M) = 0\).
\end{definition}

\begin{proposition}
Let \(M\) and \(N\) be \(R\)-modules and \(f\colon M \to N\) a homomorphism.
Then \(f(\soc(M)) \subseteq \soc(N)\).
\end{proposition}

\begin{proposition}
Let \(M\) be an \(R\)-module and \(N\) a submodule of \(M\). Then
\[
  \soc(N) = \soc(M) \cap N.
\]
\end{proposition}

\begin{proposition}
If \(M = \bigoplus_{i\in I} M_i\), then
\[
  \soc(M) = \bigoplus_{i\in I} \soc(M_i).
\]
\end{proposition}

\begin{proposition}
Let \(R\) be a ring. Then the socle \(\soc(R_R)\) is a two-sided ideal of \(R\).
\end{proposition}



% Further chapters will be added as we progress.
% Example:
% \chapter{Title of Chapter 1}
% ...

\backmatter

% REFERENCES
% As with the RO version, we can either recreate the bibliography
% manually or from a .bib file.
% \bibliographystyle{plain}
% \bibliography{bibliography}

\end{document}
